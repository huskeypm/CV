%%
%%
%%
%% NOTE: We want to maek this general, so that we cna copy/paste to multiple 
%% projects (grants, papers) etc 
%%
%%
%%


\newcommand\pmid[1]{(PMID \href{https://www.ncbi.nlm.nih.gov/pubmed/#1}{#1})}
\newcommand\arxiv[1]{(arXiv: 
\href{https://arxiv.org/abs/#1}{#1})
}


%% 
%%  Environment
%% 
\newenvironment{packed_item}{
\begin{itemize}
  \setlength{\itemsep}{1pt}
  \setlength{\parskip}{0pt}
  \setlength{\parsep}{0pt}
}{\end{itemize}}
\newenvironment{packed_enum}{
\begin{enumerate}
  \setlength{\itemsep}{1pt}
  \setlength{\parskip}{0pt}
  \setlength{\parsep}{0pt}
}{\end{enumerate}}



\newtheorem{rxnpkh}{Reaction}[section]


% BEGIN: disable/enable margin notes
%-->enable margin notes
  \newcounter{mnote}
  \setcounter{mnote}{0}
  \newcommand{\mnote}[1]{\addtocounter{mnote}{1}
    \ensuremath{{}^{\bullet\arabic{mnote}}}
    \marginpar{\footnotesize\em\color{red}\ensuremath{\bullet\arabic{mnote}}#1}}
  \let\oldmarginpar\marginpar
    \renewcommand\marginpar[1]{\-\oldmarginpar[\raggedleft\footnotesize
#1]%
    {\raggedright\footnotesize #1}}
%-->disable margin notes




%%
%% SECTIONING
%% 
%\titleformat{\subsection}[runin]{\normalfont\normalsize\bfseries\slshape}{\thesubsection.}{0.25em}{}
%\titlespacing{\subsection}{0.0em}{.5ex}{.5ex} % {left}{before}{after}[right]
%\titleformat{\subsubsection}[runin]{\normalfont\normalsize\bfseries\slshape}{\thesubsubsection.}{0.25em}{}
%\titlespacing{\subsubsection}{0.0em}{.5ex}{.5ex} % {left}{before}{after}[right]


%%
%% FIGURES 
%%
%% right wrap
\newcommand{\figright}[3]
{
%\begin{figure}
\begin{wrapfigure}{r}{0.45\textwidth}
  \vspace{-20pt}
%\begin{figure}[ht]
  \begin{center}
%\includegraphics[width=0.45\textwidth]{#1}
  \end{center}
\vspace{-20pt}
\caption{
\mynote{figshort:#2}
\label{figshort:#2}
\small #3
}
  \vspace{-10pt}
\end{wrapfigure}
%\end{figure}
}
%\renewcommand\figright[3]{\figshort{#1}{#2}{#3}}

%% Quick Figs 
\newcommand{\figsquick}[4]
{
\begin{figure}[ht]
  \begin{center}
    \includegraphics[width=#4]{#1}
  \end{center}
  \caption{
\mynote{figshort:#2}
\label{figshort:#2}
\small #3
}
\end{figure}
}

\newcommand{\figshort}[3]
{
  \figsquick{#1}{#2}{#3}{3in}
}
\newcommand{\figbig}[3]
{
  \figsquick{#1}{#2}{#3}{7in}
}

% no figs
\newcommand{\fignodisplay}[3]
{
\begin{figure}[ht]
\begin{center}
\end{center}
\caption{
\mynote{figshort:#2}
\label{figshort:#2}
\small #3
}
\end{figure}
}


% Names 
\newcommand\pnote[1]{\mnote{PKH:#1}}
\newcommand\sanote[1]{\mnote{SA:#1}}
\newcommand\aknote[1]{\mnote{AK:#1}}
\newcommand\bsnote[1]{\mnote{BS:#1}}
\newcommand\bdsnote[1]{\mnote{BDS:#1}}
%\linenumbers
%
\newcommand{\mynote}[1]{\textcolor{red}{ #1 }}


%%%% 
%%%% Toggle etc 
%%%%
\newcommand{\um}{$\mu m$}
%\renewcommand\figright[3]{\fignodisplay{#1}{#2}{#3}}
%\renewcommand\figshort[3]{\fignodisplay{#1}{#2}{#3}}
%\renewcommand\mynote[1]{}
%\renewcommand\mnote[1]{}


%%%%
%%%% COMMANDS 
%%%%%
\newcommand\textbt[1]{\textit{\textbf{#1}}}
\newcommand\lbi{\begin{packed_item}}
\newcommand\lei{\end{packed_item}}
\newcommand\lbn{\begin{packed_enum}}
\newcommand\len{\end{packed_enum}}
\newcommand\mytag[1]{\textbf{(#1.)}} % formats subfigure labels
\newcommand{\fig}{Fig.~\ref}
\newcommand\sfig[1]{(\textit{See } \fig{#1})}
\newcommand{\sect}{Sect.~\ref}
\newcommand\ssect[1]{(\textit{see} \secn{#1} )}
\newcommand{\tbl}{Table~\ref}
\newcommand\stbl[1]{(\textit{see} \tbl{#1} )}
\newcommand{\rxn}{Reaction ~\ref}
\newcommand{\eqn}{Eq.~\ref}
\newcommand{\eqns}{Eqs.~\ref}
\newcommand{\chap}{Chapter~\ref}
\newcommand{\chaps}{Chapters~\ref}
\newcommand{\app}{Appendix~\ref}


\newacronym{crest}{CREST}{Career Readiness Education in Science and Technology}
\newcommand\crest{\gls{crest}}


\newcommand{\NCX}{NCX}
\newcommand{\camk}{CaMK}
\newcommand{\PKA}{[PKA]}
\newcommand{\SB}{Shannon-Bers}
\newcommand{\LS}{Li-Smith}
\newcommand{\wt}{wild-type}
\newacronym{hpc}{HPC}{high performance computing} 
\newcommand\hpc{\gls{hpc}}
\newacronym{pmf}{PMF}{potential of mean force}
\newcommand\pmf{\gls{pmf}}
\newacronym{rd}{RD}{regulatory domain}
\newcommand\rd{\gls{rd}}
\newacronym{aid}{AID}{auto-inhibitory domain}
\newcommand\aid{\gls{aid}}
\newacronym{bd}{BD}{Brownian dynamics}
\newcommand\bd{\gls{bd}}

\newcommand{\FP}{Fokker\-Plank}
\newcommand{\SE}{Smoluchowski equation}
\newcommand{\rde}{Reaction-diffusion equation}
\newcommand{\uM}{$\mu M$}

\renewcommand{\um}{$\mu m$}



\newcommand{\DONE}{\textcolor{green}{ Done }}
\newcommand{\INPROGRESS}{\textcolor{blue}{ In progress }}
\newcommand{\POSTPONE}{\textcolor{yellow}{ Postpone }}
\newcommand{\NOTDONE}{\textcolor{red}{ Not Done }}
\newcommand{\CANCEL}{\textcolor{red}{ Cancel }}


\newcommand{\verify}{\mynote{verify}}





\newcommand{\AP}{action potential}
\newcommand{\vcycle}{v$_{cycle}$}
\newcommand{\kdcasr}{K$_{d,SR}$}
\newcommand{\kd}{K$_d$}
\newcommand{\catwo}{Ca$^{2+}$}
\newcommand{\mgtwo}{Mg$^{2+}$}
\newcommand{\nap}{Na$^{+}$}
\newcommand{\kp}{K$^{+}$}
\newcommand{\caconc}{[Ca$^{2+}$]}
\newcommand{\koff}{k$_{off}$}
\newcommand{\kon}{k$_{on}$}
\newcommand{\konl}{k$_{on,luminal}$}
\newcommand{\fluo}{Fluo-3}

\newcommand{\BAR}{$\beta$-adrenergic}
\newcommand{\uplb}{uPLB}
\newcommand{\pplb}{pPLB}


\newcommand{\PDE}{partial differential equation}
\newcommand{\Kp}{$K_{P}$}


%\newacronym{CaMKII}{Ca2+/calmodulin-dependent protein kinase II}
%{name={CaMKII},description={Ca2+/calmodulin-dependent protein kinase II}}
%\newcommand{\camk}{CaMKII}
%
%\newacronym{PKA}{Protein kinase A}
%{name={PKA},description={Protein kinase A}}
%
%\newacronym{GOF}{gain-of-function}
%{name={GOF},description={gain-of-function}}    
%\newacronym{LOF}{loss-of-function}
%{name={LOF},description={loss-of-function}}    
%
%\newacronym{RyR}{Ryanodine receptor}
%{name={RyR},description={Ryanodine receptor}}
%
\newcommand{\newacr}[2]{
  \newacronym{#1}{#2}
  {name={#1},description={#2}}
}

%\newacronym{aid}{AID}{auto-inhibitory domain}
%\newcommand\aid{\gls{aid}}
%\newacronym{bd}{BD}{Brownian dynamics}
%\newcommand\bd{\gls{bd}}
\newacronym[plural=PPIs]{ppi}{PPI}{protein-protein interactions}
\newcommand\ppi{\gls{ppi}}
\newacronym[plural=CBPs]{cbp}{CBP}{Ca-binding protein}
\newcommand\cbp{\gls{cbp}}

\newacronym{cam}{CaM}{calmodulin}
\newcommand\cam{\gls{cam}}
\newacronym{cn}{CaN}{calcineurin}
\newcommand\cn{\gls{cn}}

\newacronym{tnc}{TnC}{troponin C}
\newcommand\tnc{\gls{tnc}}
\newacronym{tn}{Tn}{troponin}
\newcommand\tn{\gls{tn}}
\newacronym{tni}{TnI}{troponin I}
\newcommand\tni{\gls{tni}}


% \newacronym{ha}{H$_{\mbox{A}}$}{helix A}
\newacronym{RMSF}{root mean squared fluctuations}
{name={RMSF},description={root mean squared fluctuations}}
\newacronym{RMSD}{root mean squared deviations}
{name={RMSD},description={root mean squared deviations}}

\newacronym{NMR}{nuclear magnetic resonance}
{name={NMR},description={nuclear magnetic resonance}}
%\newacronym{HF}{heart failure}
%{name={HF},description={heart failure}}
\newcommand{\HF}{heart failure}
\newacronym{KO}{knock-out}
{name={KO},description={knock-out}}
\newacronym{GPCR}{G-protein coupled receptor}
{name={GPCR},description={G-protein coupled receptor}}               
\newacronym{NSR}{non-junctional sarcoplasmic reticulum}
{name={NSR},description={non-junctional sarcoplasmic reticulum}}
\newacronym{SSL}{sub-sarcolemmal space}
{name={SSL},description={sub-sarcolemmal space}}             
\newacronym{JSR}{junctional sarcoplasmic reticulum}
{name={JSR},description={junctional sarcoplasmic reticulum}}
\newacronym{CVM}{cardiac ventricular myocyte}
{name={CVM},description={Cardiac ventricular myocyte}}
\newacronym{PCA}{principal components analysis}
{name={PCA},description={principal components analysis}}       
%\newacronym{BD}{Brownian dynamics}
%{name={BD},description={Brownian dynamics}}
\newacronym{TI}{thermodynamic integration}
{name={TI},description={thermodynamic integration}}
\newacronym{PLB}{phospholamban}
{name={PLB},description={phospholamban}}
\newacronym{DCM}{diabetic cardiac myopathy}
{name={DCM},description={Diabetic cardiac myopathy}}
\newacronym{CHF}{congestive heart failure}
{name={CHF},description={Congestive heart failure}}

\newacronym{LBD}{ligand-binding domain}
{name={LBD},description={Ligand-binding domain}}
\newacronym{TM}{transmembrane}
{name={TM},description={Transmembrane}}
\newacronym{EAD}{early after-depolarization}
{name={EAD},description={Early after-depolarization}}
\newcommand{\DAD}{delayed after-depolarization}
%\newacronym{DAD}{delayed after-depolarization}
%{name={DAD},description={Delayed after-depolarization}}

\newcommand\WT{wild-type}
\newacronym{nmr}{NMR}{nuclear magnetic resonance}
\newcommand\nmr{\gls{nmr}}



%\newcommand\NMR{\gls{nmr}}

\newacronym{fem}{FEM}{finite element method} 
\newcommand{\fem}{\gls{fem}}
\newacronym{remd}{REMD}{replica exchange molecular dynamics}
\newcommand{\remd}{\gls{remd}}
 
\newacronym{md}{MD}{molecular dynamics}
\newcommand\md{\gls{md}}
\newacronym{AMD}{AMD}{accelerated molecular dynamics}
\newacronym{pnp}{PNP}{Poisson-Nernst-Planck}
\newcommand\pnp{\gls{pnp}}
\newacronym{msa}{MSA}{mean spherical approximation}
\newcommand\msa{\gls{msa}}
\newacronym{gb}{GB}{Generalized Born}
\newcommand\gb{\gls{gb}}

\newacronym{fft}{FFT}{fast Fourier transform}
\newcommand\fft{\gls{fft}}



\newacronym{fret}{FRET}{Forster resonance energy transfer}
\newcommand\fret{\gls{fret}}

\newcommand{\SLN}{sarcolipin}
\newcommand{\rdf}{radial distribution function} 
\newcommand{\vmax}{$V_{max}$}
\newcommand{\dgelec}{$\Delta G_{elec}$}
\newcommand\sone{S100A1}

\newacronym{pv}{PV}{parvalbumin}
\newcommand\pv{\gls{pv}}
\newcommand\bpv{$\beta$-\pv} 
\newcommand\apv{$\alpha$-\pv} 
\newcommand\cEF{cEF} 
\newcommand\pEF{pEF} 
\newcommand{\fenics}{FEniCS}

\newcommand\hn{H$_{\mbox{N}}$}
\newacronym{ha}{H$_{\mbox{A}}$}{helix A}
\newcommand\ha{\gls{ha}}
\newacronym{hb}{H$_{\mbox{B}}$}{helix B}
\newcommand\hb{\gls{hb}}
\newacronym{hc}{H$_{\mbox{C}}$}{helix C}
\newcommand\hc{\gls{hc}}
\newacronym{hd}{H$_{\mbox{D}}$}{helix D}
\newcommand\hd{\gls{hd}}
\newacronym{he}{H$_{\mbox{E}}$}{helix E}
\newcommand\he{\gls{he}}
\newacronym{hf}{H$_{\mbox{F}}$}{helix F}
\newcommand\hf{\gls{hf}}
\newcommand\lcd{L$_{\mbox{CD}}$}
\newcommand\lef{L$_{\mbox{EF}}$}
\newcommand\calpha{C$_\alpha$}

\newcommand\addref{\mnote{add ref}}
\newcommand\addrefn[1]{\mnote{Ref #1}}


\newacronym{glu}{E}{glutamic acid}
\newcommand\glu{\gls{glu}}
\newacronym{ser}{S}{serine}
\newcommand\ser{\gls{ser}}
\newacronym{lys}{K}{lysine}
\newcommand\lys{\gls{lys}}
\newacronym{asp}{D}{aspartic acid}
\newcommand\asp{\gls{asp}}
\newacronym{tyr}{Y}{tyrosine}
\newcommand\tyr{\gls{tyr}}
\newacronym{gly}{G}{glycine}
\newcommand\gly{\gls{gly}}




\newacronym[plural={root mean square fluctuations}]{rmsf}{RMSF}{root mean squared fluctuations}
\newcommand\rmsf{\gls{rmsf}}
\newacronym{rmsd}{RMSD}{root mean squared deviations}
\newcommand\rmsd{\gls{rmsd}}

\newacronym{bscc}{BSCC}{Big Sandy Community College}
\newcommand\bscc{\gls{bscc}}

\newacronym{uk}{UK}{University of Kentucky}
\newcommand\uk{\gls{uk}}