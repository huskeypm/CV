My research is dedicated toward understanding molecular mechanisms of intracellular signaling pathways through computational modeling. 
These signaling pathways integrate molecular events, such as small molecule binding, to orchestrate myriad biological functions; oftentimes, these concerted processes are optimized to maximize speed, energy efficiency or robustness, through controlling the location, binding kinetics, and molecular composition of participating enzymes and substrates. 
My lab specifically targets calcium (\catwo) signaling in mammalian cells with a particularly emphasis on cardiac tissue, beginning with collaborative studies and grant funding during my postdoctoral studies at the University of California San Diego (UCSD) to address some of the most pressing problems in the understanding of cell biology.
Our most recent preliminary work on \catwo-regulation in two murine models, for which sarcolemmal (SL) currents are up-regulated, led us to discover the role of increased sarcoplasmic reticulum (SR) \catwo\ load in amplifying cytosolic \catwo\ transients in those respective phenotypes. 
We are currently investigating additional cellular mechanisms, by which the cardiomyocyte and its structural remodeling compensates for increased SL entry of \catwo\, as well as how to mitigate \catwo\ dysregulation through targeting specific \catwo\-handling proteins. The studies are complemented with molecular simulations of \catwo\ binding protein function as well as mesoscopic ionic transport, which we are used to understand the subcellular basis of signaling at the cellular scale. 
A unique angle to our subcellular approach is our usage of advanced computer vision techniques to process and detect features in bio-imaging data, based on my experience at the defense subcontractor Arete Associates following graduate school. 
Our long-term goal is to leverage these studies to understand how cardiac signaling pathways are controlled at the cellular level and perturbed in disease, which will prioritize molecular strategies to pharmaceutically restore normal cardiac output. 
Our manuscripts 
% 
(\href{https://www.ncbi.nlm.nih.gov/sites/myncbi/1TY9bcXrU0YAs/bibliography/43451609/public/?sort=date&direction=ascending}{MYNBCI}) 
evidence major contributions from my trainees toward this goal:
\lbn
\item Scott C and Kekenes-Huskey P, “Molecular Basis of S100A1 Activation at Saturating and Subsaturating Calcium Concentrations.,” Biophys J, vol. 110, no. 5, pp. 1052–1063, Mar. 2016.
\item Kucharski, N, Scott C, and Kekenes-Huskey P, “Understanding Ion Binding Affinity and Selectivity in β Parvalbumin Using Molecular Dynamics and Mean Sphere Approximation Theory,” J Phys Chem B, Jun. 2016.
\item Atalay S, Scott C, Kekenes-Huskey P. Quantifying the influence of the crowded cytoplasm on ion and small biomolecule diffusion via homogenization theory (in print)
\item Siddiqui, Jm Svetlana B Tikunova, Shane D Walton, Meredith Meyer, Peter P de Tombe, Nathan Neilson, Peter M Kekenes-Huskey, Hussam E Salhi, Paul M L Janssen, Brandon J Biesiadecki, Jonathan P Davis, “Myofilament Calcium Sensitivity:  Consequences of the Effective Concentration of Troponin I" (in revision)
\item Atalay, S, Scott, CE, Satin, J, Kekenes-Huskey, P ,“Microstructure within transverse tubules help control \catwo\-induced \catwo\ release in healthy and pathologically-remodeled cardiomyocytes” (in preparation)
\item Stewart, B, Scott CE, Despa, S, Despa, F Kekenes-Huskey P “Effects of amylin on cardiac calcium homeostatis” (in preparation). 
\item Cook Erik, Sun B, Kekenes-Huskey P, Creamer T “Diffusion limited association of calcineurin and calmodulin” (in preparation). 
\len

It should also be emphasized that I am committed to mentoring undergraduate students through postdoctoral scholars. 
The papers published or in preparation from my lab all have significant contributions from junior scientists at the undergraduate level and up.  
Aside from mentoring projects, we hold weekly group meetings to practice presentation skills and discuss literature, monthly joint group meetings with Dr. Christy Payne, chemical engineering, as well as frequent attendance of seminars on campus. 
I also provide travel support for lab members on at least a yearly basis to gain exposure to other labs’ research approaches and improve networking opportunities. 


