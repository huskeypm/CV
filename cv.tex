% Long Sectioned Curriculum Vitae
% LaTeX Template
%
% This template has been downloaded from:
% http://www.latextemplates.com
%
% Original author:
% Rensselaer Polytechnic Institute (http://www.rpi.edu/dept/arc/training/latex/resumes/)
%
% Important note:
% This template requires the res.cls file to be in the same directory as the
% .tex file. The res.cls file provides the resume style used for structuring the
% document.
%
%%%%%%%%%%%%%%%%%%%%%%%%%%%%%%%%%%%%%%%%%

%----------------------------------------------------------------------------------------
%	PACKAGES AND OTHER DOCUMENT CONFIGURATIONS
%----------------------------------------------------------------------------------------

\documentclass{res} % Use the res.cls style
\usepackage[normalem]{ulem}

%%%% For biosketch sections
% Initially had to compile as 'article' to get bibs to work 
%\documentclass{article}
\usepackage[T1]{fontenc}
\usepackage[colorlinks = true,
            linkcolor = blue,
            urlcolor  = blue,
            citecolor = blue,
            anchorcolor = blue]{hyperref}
%\usepackage[utf8]{inputenc}
\usepackage{xcolor}
\usepackage{lmodern}
\usepackage[english]{babel}
\usepackage[autostyle]{csquotes}
\usepackage{glossaries}
\usepackage{verbatim}
%\usepackage[sorting=none, backend=biber]{biblatex}
% Initially had to compile as 'article' to get bibs to work 
\usepackage[backend=biber,doi=false,url=false,autocite=superscript,sorting=none,firstinits=true,style=numeric-comp]{biblatex}
\addbibresource{pkh.bib}
\newcommand{\x}[1]{ } % CLUDGY HACK TO GET BIBER TO WORK ON STUPID UNICODE CHARACTERS

%%%%
\usepackage{helvet} % Default font is the helvetica postscript font
%\usepackage{newcent} % To change the default font to the new century schoolbook postscript font uncomment this line and comment the one aboveHolly
\usepackage{etaremune}
\usepackage{xkeyval}

%\newsectionwidth{0pt} % Stops section indenting

\begin{document}
%%
%%
%%
%% NOTE: We want to maek this general, so that we cna copy/paste to multiple 
%% projects (grants, papers) etc 
%%
%%
%%


\newcommand\pmid[1]{(PMID \href{https://www.ncbi.nlm.nih.gov/pubmed/#1}{#1})}
\newcommand\arxiv[1]{(arXiv: 
\href{https://arxiv.org/abs/#1}{#1})
}


%% 
%%  Environment
%% 
\newenvironment{packed_item}{
\begin{itemize}
  \setlength{\itemsep}{1pt}
  \setlength{\parskip}{0pt}
  \setlength{\parsep}{0pt}
}{\end{itemize}}
\newenvironment{packed_enum}{
\begin{enumerate}
  \setlength{\itemsep}{1pt}
  \setlength{\parskip}{0pt}
  \setlength{\parsep}{0pt}
}{\end{enumerate}}



\newtheorem{rxnpkh}{Reaction}[section]


% BEGIN: disable/enable margin notes
%-->enable margin notes
  \newcounter{mnote}
  \setcounter{mnote}{0}
  \newcommand{\mnote}[1]{\addtocounter{mnote}{1}
    \ensuremath{{}^{\bullet\arabic{mnote}}}
    \marginpar{\footnotesize\em\color{red}\ensuremath{\bullet\arabic{mnote}}#1}}
  \let\oldmarginpar\marginpar
    \renewcommand\marginpar[1]{\-\oldmarginpar[\raggedleft\footnotesize
#1]%
    {\raggedright\footnotesize #1}}
%-->disable margin notes




%%
%% SECTIONING
%% 
%\titleformat{\subsection}[runin]{\normalfont\normalsize\bfseries\slshape}{\thesubsection.}{0.25em}{}
%\titlespacing{\subsection}{0.0em}{.5ex}{.5ex} % {left}{before}{after}[right]
%\titleformat{\subsubsection}[runin]{\normalfont\normalsize\bfseries\slshape}{\thesubsubsection.}{0.25em}{}
%\titlespacing{\subsubsection}{0.0em}{.5ex}{.5ex} % {left}{before}{after}[right]


%%
%% FIGURES 
%%
%% right wrap
\newcommand{\figright}[3]
{
%\begin{figure}
\begin{wrapfigure}{r}{0.45\textwidth}
  \vspace{-20pt}
%\begin{figure}[ht]
  \begin{center}
%\includegraphics[width=0.45\textwidth]{#1}
  \end{center}
\vspace{-20pt}
\caption{
\mynote{figshort:#2}
\label{figshort:#2}
\small #3
}
  \vspace{-10pt}
\end{wrapfigure}
%\end{figure}
}
%\renewcommand\figright[3]{\figshort{#1}{#2}{#3}}

%% Quick Figs 
\newcommand{\figsquick}[4]
{
\begin{figure}[ht]
  \begin{center}
    \includegraphics[width=#4]{#1}
  \end{center}
  \caption{
\mynote{figshort:#2}
\label{figshort:#2}
\small #3
}
\end{figure}
}

\newcommand{\figshort}[3]
{
  \figsquick{#1}{#2}{#3}{3in}
}
\newcommand{\figbig}[3]
{
  \figsquick{#1}{#2}{#3}{7in}
}

% no figs
\newcommand{\fignodisplay}[3]
{
\begin{figure}[ht]
\begin{center}
\end{center}
\caption{
\mynote{figshort:#2}
\label{figshort:#2}
\small #3
}
\end{figure}
}


% Names 
\newcommand\pnote[1]{\mnote{PKH:#1}}
\newcommand\sanote[1]{\mnote{SA:#1}}
\newcommand\aknote[1]{\mnote{AK:#1}}
\newcommand\bsnote[1]{\mnote{BS:#1}}
\newcommand\bdsnote[1]{\mnote{BDS:#1}}
%\linenumbers
%
\newcommand{\mynote}[1]{\textcolor{red}{ #1 }}


%%%% 
%%%% Toggle etc 
%%%%
\newcommand{\um}{$\mu m$}
%\renewcommand\figright[3]{\fignodisplay{#1}{#2}{#3}}
%\renewcommand\figshort[3]{\fignodisplay{#1}{#2}{#3}}
%\renewcommand\mynote[1]{}
%\renewcommand\mnote[1]{}


%%%%
%%%% COMMANDS 
%%%%%
\newcommand\textbt[1]{\textit{\textbf{#1}}}
\newcommand\lbi{\begin{packed_item}}
\newcommand\lei{\end{packed_item}}
\newcommand\lbn{\begin{packed_enum}}
\newcommand\len{\end{packed_enum}}
\newcommand\mytag[1]{\textbf{(#1.)}} % formats subfigure labels
\newcommand{\fig}{Fig.~\ref}
\newcommand\sfig[1]{(\textit{See } \fig{#1})}
\newcommand{\sect}{Sect.~\ref}
\newcommand\ssect[1]{(\textit{see} \secn{#1} )}
\newcommand{\tbl}{Table~\ref}
\newcommand\stbl[1]{(\textit{see} \tbl{#1} )}
\newcommand{\rxn}{Reaction ~\ref}
\newcommand{\eqn}{Eq.~\ref}
\newcommand{\eqns}{Eqs.~\ref}
\newcommand{\chap}{Chapter~\ref}
\newcommand{\chaps}{Chapters~\ref}
\newcommand{\app}{Appendix~\ref}


\newacronym{crest}{CREST}{Career Readiness Education in Science and Technology}
\newcommand\crest{\gls{crest}}


\newcommand{\NCX}{NCX}
\newcommand{\camk}{CaMK}
\newcommand{\PKA}{[PKA]}
\newcommand{\SB}{Shannon-Bers}
\newcommand{\LS}{Li-Smith}
\newcommand{\wt}{wild-type}
\newacronym{hpc}{HPC}{high performance computing} 
\newcommand\hpc{\gls{hpc}}
\newacronym{pmf}{PMF}{potential of mean force}
\newcommand\pmf{\gls{pmf}}
\newacronym{rd}{RD}{regulatory domain}
\newcommand\rd{\gls{rd}}
\newacronym{aid}{AID}{auto-inhibitory domain}
\newcommand\aid{\gls{aid}}
\newacronym{bd}{BD}{Brownian dynamics}
\newcommand\bd{\gls{bd}}

\newcommand{\FP}{Fokker\-Plank}
\newcommand{\SE}{Smoluchowski equation}
\newcommand{\rde}{Reaction-diffusion equation}
\newcommand{\uM}{$\mu M$}

\renewcommand{\um}{$\mu m$}



\newcommand{\DONE}{\textcolor{green}{ Done }}
\newcommand{\INPROGRESS}{\textcolor{blue}{ In progress }}
\newcommand{\POSTPONE}{\textcolor{yellow}{ Postpone }}
\newcommand{\NOTDONE}{\textcolor{red}{ Not Done }}
\newcommand{\CANCEL}{\textcolor{red}{ Cancel }}


\newcommand{\verify}{\mynote{verify}}





\newcommand{\AP}{action potential}
\newcommand{\vcycle}{v$_{cycle}$}
\newcommand{\kdcasr}{K$_{d,SR}$}
\newcommand{\kd}{K$_d$}
\newcommand{\catwo}{Ca$^{2+}$}
\newcommand{\mgtwo}{Mg$^{2+}$}
\newcommand{\nap}{Na$^{+}$}
\newcommand{\kp}{K$^{+}$}
\newcommand{\caconc}{[Ca$^{2+}$]}
\newcommand{\koff}{k$_{off}$}
\newcommand{\kon}{k$_{on}$}
\newcommand{\konl}{k$_{on,luminal}$}
\newcommand{\fluo}{Fluo-3}

\newcommand{\BAR}{$\beta$-adrenergic}
\newcommand{\uplb}{uPLB}
\newcommand{\pplb}{pPLB}


\newcommand{\PDE}{partial differential equation}
\newcommand{\Kp}{$K_{P}$}


%\newacronym{CaMKII}{Ca2+/calmodulin-dependent protein kinase II}
%{name={CaMKII},description={Ca2+/calmodulin-dependent protein kinase II}}
%\newcommand{\camk}{CaMKII}
%
%\newacronym{PKA}{Protein kinase A}
%{name={PKA},description={Protein kinase A}}
%
%\newacronym{GOF}{gain-of-function}
%{name={GOF},description={gain-of-function}}    
%\newacronym{LOF}{loss-of-function}
%{name={LOF},description={loss-of-function}}    
%
%\newacronym{RyR}{Ryanodine receptor}
%{name={RyR},description={Ryanodine receptor}}
%
\newcommand{\newacr}[2]{
  \newacronym{#1}{#2}
  {name={#1},description={#2}}
}

%\newacronym{aid}{AID}{auto-inhibitory domain}
%\newcommand\aid{\gls{aid}}
%\newacronym{bd}{BD}{Brownian dynamics}
%\newcommand\bd{\gls{bd}}
\newacronym[plural=PPIs]{ppi}{PPI}{protein-protein interactions}
\newcommand\ppi{\gls{ppi}}
\newacronym[plural=CBPs]{cbp}{CBP}{Ca-binding protein}
\newcommand\cbp{\gls{cbp}}

\newacronym{cam}{CaM}{calmodulin}
\newcommand\cam{\gls{cam}}
\newacronym{cn}{CaN}{calcineurin}
\newcommand\cn{\gls{cn}}

\newacronym{tnc}{TnC}{troponin C}
\newcommand\tnc{\gls{tnc}}
\newacronym{tn}{Tn}{troponin}
\newcommand\tn{\gls{tn}}
\newacronym{tni}{TnI}{troponin I}
\newcommand\tni{\gls{tni}}


% \newacronym{ha}{H$_{\mbox{A}}$}{helix A}
\newacronym{RMSF}{root mean squared fluctuations}
{name={RMSF},description={root mean squared fluctuations}}
\newacronym{RMSD}{root mean squared deviations}
{name={RMSD},description={root mean squared deviations}}

\newacronym{NMR}{nuclear magnetic resonance}
{name={NMR},description={nuclear magnetic resonance}}
%\newacronym{HF}{heart failure}
%{name={HF},description={heart failure}}
\newcommand{\HF}{heart failure}
\newacronym{KO}{knock-out}
{name={KO},description={knock-out}}
\newacronym{GPCR}{G-protein coupled receptor}
{name={GPCR},description={G-protein coupled receptor}}               
\newacronym{NSR}{non-junctional sarcoplasmic reticulum}
{name={NSR},description={non-junctional sarcoplasmic reticulum}}
\newacronym{SSL}{sub-sarcolemmal space}
{name={SSL},description={sub-sarcolemmal space}}             
\newacronym{JSR}{junctional sarcoplasmic reticulum}
{name={JSR},description={junctional sarcoplasmic reticulum}}
\newacronym{CVM}{cardiac ventricular myocyte}
{name={CVM},description={Cardiac ventricular myocyte}}
\newacronym{PCA}{principal components analysis}
{name={PCA},description={principal components analysis}}       
%\newacronym{BD}{Brownian dynamics}
%{name={BD},description={Brownian dynamics}}
\newacronym{TI}{thermodynamic integration}
{name={TI},description={thermodynamic integration}}
\newacronym{PLB}{phospholamban}
{name={PLB},description={phospholamban}}
\newacronym{DCM}{diabetic cardiac myopathy}
{name={DCM},description={Diabetic cardiac myopathy}}
\newacronym{CHF}{congestive heart failure}
{name={CHF},description={Congestive heart failure}}

\newacronym{LBD}{ligand-binding domain}
{name={LBD},description={Ligand-binding domain}}
\newacronym{TM}{transmembrane}
{name={TM},description={Transmembrane}}
\newacronym{EAD}{early after-depolarization}
{name={EAD},description={Early after-depolarization}}
\newcommand{\DAD}{delayed after-depolarization}
%\newacronym{DAD}{delayed after-depolarization}
%{name={DAD},description={Delayed after-depolarization}}

\newcommand\WT{wild-type}
\newacronym{nmr}{NMR}{nuclear magnetic resonance}
\newcommand\nmr{\gls{nmr}}



%\newcommand\NMR{\gls{nmr}}

\newacronym{fem}{FEM}{finite element method} 
\newcommand{\fem}{\gls{fem}}
\newacronym{remd}{REMD}{replica exchange molecular dynamics}
\newcommand{\remd}{\gls{remd}}
 
\newacronym{md}{MD}{molecular dynamics}
\newcommand\md{\gls{md}}
\newacronym{AMD}{AMD}{accelerated molecular dynamics}
\newacronym{pnp}{PNP}{Poisson-Nernst-Planck}
\newcommand\pnp{\gls{pnp}}
\newacronym{msa}{MSA}{mean spherical approximation}
\newcommand\msa{\gls{msa}}
\newacronym{gb}{GB}{Generalized Born}
\newcommand\gb{\gls{gb}}

\newacronym{fft}{FFT}{fast Fourier transform}
\newcommand\fft{\gls{fft}}



\newacronym{fret}{FRET}{Forster resonance energy transfer}
\newcommand\fret{\gls{fret}}

\newcommand{\SLN}{sarcolipin}
\newcommand{\rdf}{radial distribution function} 
\newcommand{\vmax}{$V_{max}$}
\newcommand{\dgelec}{$\Delta G_{elec}$}
\newcommand\sone{S100A1}

\newacronym{pv}{PV}{parvalbumin}
\newcommand\pv{\gls{pv}}
\newcommand\bpv{$\beta$-\pv} 
\newcommand\apv{$\alpha$-\pv} 
\newcommand\cEF{cEF} 
\newcommand\pEF{pEF} 
\newcommand{\fenics}{FEniCS}

\newcommand\hn{H$_{\mbox{N}}$}
\newacronym{ha}{H$_{\mbox{A}}$}{helix A}
\newcommand\ha{\gls{ha}}
\newacronym{hb}{H$_{\mbox{B}}$}{helix B}
\newcommand\hb{\gls{hb}}
\newacronym{hc}{H$_{\mbox{C}}$}{helix C}
\newcommand\hc{\gls{hc}}
\newacronym{hd}{H$_{\mbox{D}}$}{helix D}
\newcommand\hd{\gls{hd}}
\newacronym{he}{H$_{\mbox{E}}$}{helix E}
\newcommand\he{\gls{he}}
\newacronym{hf}{H$_{\mbox{F}}$}{helix F}
\newcommand\hf{\gls{hf}}
\newcommand\lcd{L$_{\mbox{CD}}$}
\newcommand\lef{L$_{\mbox{EF}}$}
\newcommand\calpha{C$_\alpha$}

\newcommand\addref{\mnote{add ref}}
\newcommand\addrefn[1]{\mnote{Ref #1}}


\newacronym{glu}{E}{glutamic acid}
\newcommand\glu{\gls{glu}}
\newacronym{ser}{S}{serine}
\newcommand\ser{\gls{ser}}
\newacronym{lys}{K}{lysine}
\newcommand\lys{\gls{lys}}
\newacronym{asp}{D}{aspartic acid}
\newcommand\asp{\gls{asp}}
\newacronym{tyr}{Y}{tyrosine}
\newcommand\tyr{\gls{tyr}}
\newacronym{gly}{G}{glycine}
\newcommand\gly{\gls{gly}}




\newacronym[plural={root mean square fluctuations}]{rmsf}{RMSF}{root mean squared fluctuations}
\newcommand\rmsf{\gls{rmsf}}
\newacronym{rmsd}{RMSD}{root mean squared deviations}
\newcommand\rmsd{\gls{rmsd}}

\newacronym{bscc}{BSCC}{Big Sandy Community College}
\newcommand\bscc{\gls{bscc}}

\newacronym{uk}{UK}{University of Kentucky}
\newcommand\uk{\gls{uk}}
% For CV 
%----------------------------------------------------------------------------------------
%	YOUR NAME AND ADDRESS(ES) SECTION
%----------------------------------------------------------------------------------------

\name{Peter M. Kekenes-Huskey, Ph.D.\\ \\} % Your name at the top

% If you don't want one of the addresses, simply remove all the text in the first or second \address{} bracket
\newcommand{\mytilde}{\raise.17ex\hbox{$\scriptstyle\mathtt{\sim}$}}
\address{\\%{\bf School Address} \\ 
Department of Chemistry \\ 
University of Kentucky\\
Lexington, KY 40506  } % Your address 1

\address{\\%{\bf Contact} \\ 
pkekeneshuskey@uky.edu \\ http://pkh.as.uky.edu \\  (859) 323-1573} % Your address 2
% Testing this new change
%----------------------------------------------------------------------------------------
\begin{resume}

%%----------------------------------------------------------------------------------------
%%	OBJECTIVE SECTION
%%----------------------------------------------------------------------------------------
%
%\section{\centerline{OBJECTIVE}}
%
%\vspace{8pt} % Gap between title and text
%
%A position in Personnel Administration utilizing skills in recruiting, training and compensation.\\ 

%----------------------------------------------------------------------------------------
%	EDUCATION SECTION
%----------------------------------------------------------------------------------------

\section{\centerline{EDUCATION}} 

\vspace{8pt} % Gap between title and text

{\sl Doctorate of Philosophy}, 
Chemistry  \\ 
California Institute of Technology, Pasadena, CA \hfill  Spring 2009 \\ 
%(GPA: 4.0 in major, 3.40 overall)
 
{\sl Bachelor of Science}, Chemistry \\ 
University of North Carolina, Asheville, NC \hfill May 2001 \\
\emph{Summa Cum Laude} %(GPA: 4.0)

%----------------------------------------------------------------------------------------
 
\vspace{0.2in} % Some whitespace between sections

%----------------------------------------------------------------------------------------
%	PROFESSIONAL EXPERIENCE SECTION
%----------------------------------------------------------------------------------------

\section{\centerline{PROFESSIONAL EXPERIENCE}} 

\vspace{8pt} % Gap between title and text

{\sl University of Kentucky, Lexington, KY } \hfill 2014 - present\\ 
Assistant Professor of Chemistry\\
Secondary Graduate Faculty in Chemical and Materials Engineering
%\lbi
%\item Multiscale modeling of cardiac function and diffusional processes
%\lei


{\sl University of California San Diego,  San
Diego CA}[JA
McCammon, AD McCulloch] \hfill 2010 - 2014\\ 
Postdoctoral fellow 
%\lbi
%\item Multiscale modeling of cardiac function and diffusional processes
%\lei

{\sl Arete Associates, Staff Scientist, Northridge CA} \hfill 	2007 - 2010\\
Staff Scientist
%\lbi
%\item Signal processing, detection theory, algorithm design 
%\lei

{\sl Sandia National Laboratory, Albuquerque, NM. } [PS Crozier] \hfill 	summer 2005\\
Summer Internship
%\lbi
%\item Free energy calculations for ion diffusion in silica nanopores
%\lei

{\sl California Institute of Technology, Pasadena, CA.}  [WA Goddard, III]	  \hfill 2001 - 2007\\
Graduate Student
%\lbi
%\item Drug design algorithm development and application
%\lei

{\sl Freie Universitaet zu Berlin, Berlin, Germany. } [EW Knapp] \hfill 2001 - 2002\\
Fulbright fellow
%\lbi
%\item Molecular dynamics simulation of estrogen derivative binding to the estrogen receptor
%\lei

{\sl U. North Carolina, Asheville, NC. } [G Heard, BE Holmes]	 \hfill 1999 - 2001\\
Undergraduate researcher
%\lbi
%\item Quantum chemistry calculations of chlorofluorocarbon decomposition kinetics
%\lei

{\sl University of Cincinnati, OH. } [T Beck, W Connick] \hfill 	summer 2000 \\
Summer researcher 
%\lbi
%\item Quantum chemistry calculations of electron transfer processes
%\lei


%{\sl International Business Machines} \hfill January -- August 1990 \\
%General Products Division, Tucson, AZ \hfill (Co-op Assignment)
%\begin{itemize} \itemsep -2pt % Reduce space between items
%\item Developed new selection criteria for applicant screening and selection at GPD Tucson. Conducted job analysis, wrote criteria, identified skill codes for applicant tracking, established rater reliability. 
%\item Interviewed applicants for positions in Assembly, Warehousing, and Direct Customer Response. 
%\item Assistant Co-op Coordinator. Initiated and maintained computer tracking for Co-op program. Organized all co-op seminars and activities, co-op directory. 
%\item Representative on GPD Compensation Task Force. Prepared job descriptions, assigned corporate position code, and submitted for division approval. 
%\end{itemize}

%----------------------------------------------------------------------------------------

\vspace{0.2in} % Some whitespace between sections

%----------------------------------------------------------------------------------------
%	COMPUTER SKILLS SECTION
%----------------------------------------------------------------------------------------
\newpage
\section{\centerline{RESEARCH EMPHASIS}}

\vspace{8pt} % Gap between title and text

{\sl Computational chemistry and biophysics}
\begin{itemize} \itemsep -2pt
\item Systems modeling of calcium signaling in cardiac and other eukaryotic cells
\item Molecular dynamics modeling of regulatory proteins, finite element modeling of small molecule transport
\item \href{https://www.ncbi.nlm.nih.gov/pubmed/?term=kekenes-huskey}{Pubmed} 
and \href{https://scholar.google.com/citations?user=2W2u-KsAAAAJ&hl=en}{Google Scholar}
\end{itemize}

%----------------------------------------------------------------------------------------

\vspace{0.2in} % Some whitespace between sections

%----------------------------------------------------------------------------------------
%	PUBLICATIONS SECTION
%----------------------------------------------------------------------------------------

\section{\centerline{AWARDS}} 

\vspace{8pt} % Gap between title and text



{\sl Faculty}
\begin{itemize} \itemsep -12pt % Reduce space between items
\item  Nominee for \acrfull{uk} Faculty Mentor of the Year \hfill 2018 \\
\item  UK Office of Undergraduate Research's Faculty Mentor of the Week \hfill 2018 \\
\item Doctoral New Investigator Grant from the American Chemical Society \hfill 2017\\
\item UK Arts\&Sciences Award for Innovative Teaching \hfill 2017\\
\item Recognized as "Teacher who made a difference" (UK) \hfill 2016\\
\item UK Nominee for Blavatnik National Awards for Young Scientists \hfill 2016-17\\
\item UK Nominee for 2016 Simon's Investigator of Math
Modeling of Living Systems award \hfill 2015 \\
\end{itemize}

{\sl Post-graduate}
\begin{itemize} \itemsep -12pt % Reduce space between items
\item National Institutes of Health Ruth Kirschstein Postdoctoral Fellow  \hfill 2013 \\
\item American Heart Association Western States Affiliates Postdoctoral
Fellow \hfill 2013 \\
\item Vice President Discretionary Award (Arete Associates)	\hfill 2010 \\
\end{itemize}

{\sl Graduate }
\begin{itemize} \itemsep -12pt % Reduce space between items
\item DOE Computational Science Graduate Fellow	\hfill 2004-2006  \\
\item National Science Foundation Fellow (declined for CSGF, 2003)	\hfill 2002-2003 \\
\item Department of Defense Fellowship (declined for NSF)  \hfill 2001 \\
\item Fulbright Fellow (Germany)	\hfill 2001 \\
\end{itemize}

{\sl Undergraduate}
\begin{itemize} \itemsep -12pt % Reduce space between items
\item Manly Wright Award, Valedictorian of Graduating Class	\hfill 2001 \\
\item Outstanding Senior of the Western Carolinas American Chemical Society	\hfill 2001 \\
\item USA Today All-Academic 3rd Team	\hfill 2001 \\
\item Barry Goldwater Scholar in Science and Mathematics	\hfill 2000-2001 \\
\item W. Carolina ACS Schweizerhalle Scholarship	\hfill 2000-2001 \\
\item Phi Eta Sigma National Honors Fraternity	\hfill 1998 \\
\item Albina Mills Academic Scholarship	\hfill 1998-2001 \\
\item 4th Place UNCA Olivia-Jones Freshman Creative Writing Contest	\hfill 1999 \\
\end{itemize}

{\sl High School}
\begin{itemize} \itemsep -12pt % Reduce space between items
\item Eagle Scout	\hfill 1998 \\
\item Lion's Eye Bank Scholarship for Post-Secondary Education	\hfill 1998 \\
\item Presidential Scholar	\hfill 1998 \\
\item National Honors Society	\hfill 1997 \\
\end{itemize}

\vspace{0.2in} % Some whitespace between sections
%----------------------------------------------------------------------------------------
%	PUBLICATIONS SECTION
%----------------------------------------------------------------------------------------
\newpage
%section{\centerline{PUBLICATIONS}}
%To gen. list 
%\documentclass{article}
%\begin{document}
%\autocite{KekenesHuskey:l_olPRe3,Setny:2013dw,Kekenes-Huskey2012c,Kekenes-Huskey2012b,Kekenes-Huskey2012a,Hake2012,Lindert2012a,Kekenes-Huskey2012,Cheng:vn,Lindert:IuL24PVU,Kekenes-Huskey2011,KekenesHuskey:2009wh,Heo:2007ty,Ferguson:2005cf,Vaidehi:2005ue,KekenesHuskey:2004va,KekenesHuskey:2003uf,Kekenes-Huskey2014a,Liao:ARrpx_SV,Hake:q7b0C29I,Eun2013}

%\bibliographystyle{jbact}
%\bibliography{/home/huskeypm/notes/pkh}
%\end{document}

% how to create
% ALL ENTRIES MUST BE ON SINGLE LINE FOR OzthER SCRIPTS TO WORK 
% cut and paste new entries as references from Papers
% replace [1] with \item
% replace Kekenes-Huskey etc with \pmkh
% add co-first with *
% Be sure to add PMID via \pmid{} command 
\newcommand\ug{$^+$}
\vspace{15pt} % Gap between title and text
\section{\centerline{PUBLICATIONS}}
\newcommand{\pmkh}{{\bf P.M. Kekenes-Huskey}}\vspace{10pt}

\centerline{\sl * equal contribution,  \ug\ undergraduate author}
\vspace{10pt}
\newcommand\asfac{10}
\newcommand\total{35} % total - under review 
\centerline{ \uline{(\asfac/\total\ as faculty member)}}
\begin{etaremune} \itemsep 2pt % Reduce space between items
\item Shen, X., Brink, J. van den, Hou, Y., Colli, D., Le, C., Kolstad, T. R., \pmkh, Louch, W. E. (2018). 3D dSTORM imaging reveals novel detail of ryanodine receptor localization in rat cardiac myocytes. The Journal of Physiology. (in print)
\item \textit{P. Wagh, X. Zhang, R. Blood\ug, \pmkh, P. Rajapaksha, Y. Wei and I. Escobar, 
"Alignment and Immobilization of Aquaporins on Polybenzimidazole Nanofiltration Membranes" (under review)}
\item \textit{Dylan Colli\ug\ et al, \pmkh,
"GPU accelerated detection and classication of myocyte transverse tubule ultra-structure",
(\href{https://www.biorxiv.org/content/early/2018/07/17/371328}{bioarxiv 371328}, under revison)}
\item Sun, B., Cook, E. C., Creamer, T. P., and \pmkh (2018). Electrostatic control of calcineurin's intrinsically-disordered regulatory domain binding to calmodulin. Biochimica et Biophysica Acta (BBA) - General Subjects, 1862(12), 2651-2659. \pmid{30071273}
\item Byeongjae Chun, Bradley D Stewart, Darin D Vaughan\ug\, Adam S Bachstetter, \pmkh,
"Simulation of P2X-mediated calcium signaling in microglia", 
The Journal of Physiology, 
(\href{https://www.biorxiv.org/content/early/2018/06/24/354142}{354142}, in print)
\item Bin Sun, Ryan Blood\ug\, Selcuk Atalay, Dylan Colli\ug\, Stephen E. Rankin, Barbara L. Knutson and \pmkh, "Simulation-based characterization of electrolyte and small molecule diffusion in imaged oriented mesoporous silica thin films", 
(chemRxiv: \href{https://doi.org/10.26434/chemrxiv.5533066.v1}{5533066}) (in press)
\item Stewart, B. D., Scott, C. E., McCoy, T. P., Yin, G., Despa, F., Despa, S., and \pmkh. (2018). "Computational modeling of amylin-induced calcium dysregulation in rat ventricular cardiomyocytes." Cell Calcium, 71, 65-74. \pmid{29604965}
\item \textit{E. C. Cook, B. Sun, \pmkh\ and T.P. Creamer,"Electrostatic Forces Mediate Fast Association of Calmodulin and the Intrinsically Disordered Regulatory Domain of Calcineurin." 2016. \arxiv{1611.04080} (under revision)}
\item JK Siddiqui, SB Tikunova, SD Walton, M Meyer, PP de Tombe, N Neilson, \pmkh, HE Salhi, PML Janssen, BJ Biesiadecki, JP Davis, "Myofilament Calcium Sensitivity:  Consequences of the Effective Concentration of Troponin I," Frontiers in Physiology, 2016, 7:632. \pmid{28066265}
\item A.N. Kucharski$^+$, C.E. Scott, J.P. Davis and \pmkh, "Understanding Ion Binding Affinity and Selectivity in $\beta$ Parvalbumin Using Molecular Dynamics and Mean Sphere Approximation Theory," J Phys Chem B, 2016, 120(33):8617-30 \pmid{28066265}
\item \pmkh, C. E. Scott, and S. Atalay, "Quantifying the influence of the crowded cytoplasm on ionic diffusion," J Phys Chem B 2016, 120(33):8696-706 \pmid{27327486}
\item C. E. Scott and \pmkh, "Molecular basis of calcium-induced structural changes of human S100A1," Biophys J, Mar. 2016, 110(5):1052-1063 \pmid{26958883}
\item \pmkh, C. Eun, and A. McCammon, "Enzyme localization, crowding, and buffers collectively modulate diffusion-influenced signal transduction: Insights from continuum diffusion modeling," Journal of Chemical Physics, 2015, 143(9):1-12. \pmid{26342355}
\vspace{20pt}
\\
%\end{etaremune}
\centerline{ (25/\total\ up through postdoctoral studies)}
\vspace{-10pt}
%\begin{etaremune}
%\setcounter{enumi}{11} % value of last entry
\item S. Lindert, Y. Cheng, \pmkh, M. Regnier, and J. A. McCammon, "Effects of HCM cTnI mutation R145G on troponin structure and modulation by PKA phosphorylation elucidated by molecular dynamics simulations.," Biophys J, vol. 108, no. 2, pp. 395-407, Jan. 2015. \pmid{25606687}
\item N. Wang, S. Zhou, \pmkh, B. Li, and J. A. McCammon, "Poisson-Boltzmann vs. Size-modified Poisson-Boltzmann Electrostatics Applied to Lipid Bilayers," J Phys Chem B, p. 141126142529007, Nov. 2014. \pmid{25426875}
\item V. T. Metzger, C. Eun, \pmkh, G. Huber, and J. A. McCammon, "Electrostatic Channeling in P. falciparum DHFR-TS: Brownian Dynamics and Smoluchowski Modeling," Biophys J, vol. 107, no. 10, pp. 2394-2402, Nov. 2014. \pmid{25418308}
\item Y. Cheng, S. Lindert, \pmkh, V. S. Rao, R. J. Solaro, P. R. Rosevear, R. Amaro, A. D. Mcculloch, J. A. McCammon, and M. Regnier, "Computational Studies of the Effect of the S23D/S24D Troponin I Mutation on Cardiac Troponin Structural Dynamics," Biophys J, vol. 107, no. 7, pp. 1675-1685, Oct. 2014.\pmid{25296321}
\item \pmkh, A. K. Gillette, and J. A. McCammon, "Predicting the influence of long-range molecular interactions on macroscopic-scale diffusion by homogenization of the Smoluchowski equation," The Journal of chemical physics, vol. 140, no. 17, p. 174106, May 2014.\pmid{23293662}
\item J. Hake, \pmkh, and A. D. Mcculloch, "Computational modeling of subcellular transport and signaling," Current Opinion in Structural Biology, vol. 25, pp. 92-97, Apr. 2014.\pmid{24509246}
\item C. Eun, \pmkh*, V. T. Metzger, and J. A. McCammon, "A model study of sequential enzyme reactions and electrostatic channeling.," Journal of Chemical Physics, vol. 140, no. 10, pp. 105101-105101, Mar. 2014.\pmid{24628210}
\item \pmkh, T. Liao, A. K. Gillette, J. E. Hake, Y. Zhang, A. P. Michailova, A. D. Mcculloch, and J. A. McCammon, "Molecular and subcellular-scale modeling of nucleotide diffusion in the cardiac myofilament lattice.," Biophys J, vol. 105, no. 9, pp. 2130-2140, Nov. 2013.\pmid{24209858}
\item T. Liao, Y. Zhang, \pmkh, Y. Cheng, A. Michailova, A. D. McCulloch, M. Holst, and J. Mccammon, "Multi-core CPU or GPU-accelerated Multiscale Modeling for Biomolecular Complexes," Molecular Based , pp. 164-179, Oct. 2013.\pmid{24352481}
\item C. Eun, \pmkh, and J. A. McCammon, "Influence of neighboring reactive particles on diffusion-limited reactions.," Journal of Chemical Physics, vol. 139, no. 4, pp. 044117-044117, Jul. 2013.\pmid{23901970}
\item P. Setny, R. Baron, \pmkh, J. A. McCammon, and J. Dzubiella, "Solvent fluctuations in hydrophobic cavity-ligand binding kinetics," Proc Natl Acad Sci USA, vol. 110, no. 4, pp. 1197-1202, Jan. 2013.\pmid{23297241}
\item \pmkh, S. Lindert, and J. McCammon, "Molecular basis of calcium-sensitizing and desensitizing mutations of the human cardiac troponin C regulatory domain: a multi-scale simulation study.," PLOS Computational Biology, vol. 8, no. 11, pp. e1002777-e1002777, Nov. 2012.\pmid{23209387}
\item \pmkh*, V. Metzger*, B. Grant, and J. McCammon, "Calcium binding and allosteric signaling mechanisms for the sarcoplasmic reticulum Ca(2+)  ATPase.," Protein Sci., vol. 21, no. 10, pp. 1429-1443, Oct. 2012.\pmid{22821874}
\item J. Hake, A. G. Edwards, Z. Yu, \pmkh, A. P. Michailova, J. A. McCammon, M. J. Holst, M. Hoshijima, and A. D. Mcculloch, "Modelling cardiac calcium sparks in a three-dimensional reconstruction of a calcium release unit.," The Journal of Physiology, vol. 590, no. 18, pp. 4403-4422, Sep. 2012.\pmid{22495592}
\item S. Lindert, \pmkh, G. Huber, L. Pierce, and J. McCammon, "Dynamics and calcium association to the N-terminal regulatory domain of human cardiac troponin C: a multiscale computational study.," J Phys Chem B, vol. 116, no. 29, pp. 8449-8459, Jul. 2012.\pmid{22329450}
\item \pmkh, Y. Cheng, J. Hake, F. Sachse, J. Bridge, M. Holst, A. McCulloch, J. McCammon, and A. Michailova, "Modeling effects of L-type ca(2+) current and na(+)-ca(2+) exchanger on ca(2+) trigger flux in rabbit myocytes with realistic T-tubule geometries.," Front Physiol, vol. 3, pp. 351-351, Jan. 2012.\pmid{23060801}
\item Y. Cheng, \pmkh, J. E. Hake, M. J. Holst, J. A. McCammon, and A. P. Michailova, "Multi-scale continuum modeling of biological processes: from molecular electro-diffusion to sub-cellular signaling transduction," Comput Sci Discov, vol. 5, no. 1, p. 015002, 2012.\pmid{23505398}
\item \pmkh, A. Gillette, J. Hake, and J. A. McCammon, "Finite-element estimation of protein-ligand association rates with post-encounter effects: applications to calcium binding in troponin C and SERCA," Comput Sci Discov, vol. 5, no. 1, p. 014015, 2012.\pmid{23293662}
\item S. Lindert, \pmkh, and J. A. McCammon, "Long-Timescale Molecular Dynamics Simulations Elucidate the Dynamics and Kinetics of Exposure of the Hydrophobic Patch in Troponin C," Biophys J, vol. 103, no. 8, pp. 1784-1789, 2012. \pmid{23083722}
%\item \pmkh, Y. Cheng, and J. Hake, "Contributions of Structural t-Tubule Heterogeneities and Membrane Ca2+ Flux Localization to Local Ca2+ Signaling in Rabbit Ventricular Myocytes," Biophysical , 2011.
\item \pmkh, A Monte Carlo-based torsion construction algorithm for ligand design. Doctoral Thesis, 2009. 
\item J. Heo, S. Han, N. Vaidehi, J. Wendel, \pmkh, and W. Goddard III, "Prediction of the 3D Structure of FMRF?amide Neuropeptides Bound to the Mouse MrgC11 GPCR and Experimental Validation," ChemBioChem, vol. 8, no. 13, pp. 1527-1539, 2007.\pmid{17647204}
\item J. D. Ferguson, N. L. Johnson, \pmkh, W. C. Everett, G. L. Heard, D. W. Setser, and B. E. Holmes, "Unimolecular Rate Constants for HX or DX Elimination (X = F, Cl) from Chemically Activated CF 3CH 2CH 2Cl, C 2H 5CH 2Cl, and C 2D 5CH 2Cl:  Threshold Energies for HF and HCl Elimination," J. Phys. Chem. A, vol. 109, no. 20, pp. 4540-4551, May 2005.\pmid{16833790}
\item A. E. Cho, J. A. Wendel, N. Vaidehi, \pmkh, W. B. Floriano, P. K. Maiti, and W. A. Goddard, "The MPSim-Dock hierarchical docking algorithm: Application to the eight trypsin inhibitor cocrystals," J Comput Chem, vol. 26, no. 1, pp. 48-71, 2004.\pmid{15529328}
\item \pmkh, I. Muegge, and M. Rauch, "A molecular docking study of estrogenically active compounds with 1, 2-diarylethane and 1, 2-diarylethene pharmacophores," Bioorganic\& medicinal, 2004.\pmid{15556769}
\item \pmkh, N. Vaidehi, W. B. Floriano, and W. Goddard III, "Fidelity of phenylalanyl-tRNA synthetase in binding the natural amino acids," J Phys Chem B, vol. 107, no. 41, pp. 11549-11557, 2003. % Not in Pubmed

\end{etaremune}




%----------------------------------------------------------------------------------------

\vspace{0.2in} % Some whitespace between sections

%----------------------------------------------------------------------------------------
%	FUNDING SECTION
%----------------------------------------------------------------------------------------
\newpage
%section{\centerline{FUNDING}} 
\section{\centerline{FUNDING}} 
\newcommand{\dapi}{$\dag$}
\newcommand{\dacopi}{$^\circ$}
\newcommand{\dacoi}{$^*$}
\newcommand{\dasig}{$^+$}
{\sl \dapi principal investigator}
{\sl \dacopi co-principal investigator}
{\sl \dacoi co-investigator} 
{\sl \dasig significant contributions} 
\vspace{-10pt} % Reduce space between section title and contents

%\subsection{Pending}




(Kekenes-Huskey)\dapi \hfill 09/01/18-08/31/21 (0.40 calendar month)\\
NSF CTMC \hfill \$580,757\\
"CAREER: Unraveling calcium/calmodulin-dependent calcineurin activation via computation"\\
The major goals of this project is to examine aspects of calcium signaling in nanoscale materials. \\

(Brehm, Kekenes-Huskey)\dacopi \hfill 06/01/19-05/31/20 (0.40 calendar month)\\
NASA EPSCoR \hfill \$40,000\\
"Development of a RANS-Based Wall-Model for CartesianGrid Navier-Stokes Solvers"\\
The major goal of this project is to develop an advanced PDE solver amenable to fluid dynamics simulations  \\

%NSF MCB\dacopi \hfill 09/01/17-08/31/21 \\
%"Role of electrostatics in calcineurin activation"\\
%The major goal of this project is to understand the interplay between electrostatics and conformational dynamics in protein/protein interactions\\
%Total: \$711,975 (2.0 calendar month)
% Copied from overleaf cv; merged in completed
\subsection{Active}
\vspace{-10pt}

%%%%%%%%%%%%%%%%%%%%
1 R35 GM124977 (Kekenes-Huskey)\dapi \hfill 09/01/17-08/31/22 (1.89 calendar month) \\
NIH/NIGMS \hfill \$1,558,386.00 (incl. indirect) \\
"Probing cellular intracellular calcium signaling and sensing through computation"\\
The major goals of this project is to develop multi-scale tools to predict intracellular calcium signaling, from single molecules to the cell. %\\
%PKH: \$1,185,641/Total: \$1,185,641 
%%%%%%%%%%%%%%%%5


%%%%%%%%%%%%%%%%%%%%%%%
Petroleum Research Fund (Kekenes-Huskey)\dapi   \hfill 01/01/18-12/31/19 (TBD calendar month)\\
American Chemical Society \hfill \$110,000 (incl. indirects) \\
"Multi-Scale Modeling of Methane Permeation in Defect-Containing Zeolitic Materials"\\
Major goals include developing multi-physical, multi-scale models of gaseous substrates in highly-structured, zeolitic materials. 


5	U01	HL133359	02 (Campbell)\dasig\hfill 08/03/2018-07/31/22\\
NIH/NIGMS \hfill (\$610,274) \\
`Multiscale modeling of inherited cardiomyopathies and therapeutic interventions'\\
The major goal of this project is to create multi-scale models of cardiac function and myopathies, from the molecular to whole-organ levels.  
PKH provides molecular simulation expertise but does not currently draw funds from this award.

%%%%%%%%%%%%%%%%%%%%%%5





\subsection{Completed} 
\vspace{-10pt}

%%%%%%%%%%%%%%%%%%%%%%%
4 P20 GM103527	09 (Cassis)\dacoi \hfill 09/01/17-08/31/20,  (1.67 calendar months) \\
NIH/NIGMS\hfill \$2,257,498\\
50,000 Pilot Support through "Center of Biomedical Research Excellence (COBRE) on Obesity and Cardiovascular Diseases (COCVD)\\
The major goal of this project is to enhance the competitiveness of junior faculty with research programs. 
PKH lab supported through a 50K pilot award. % \\
%\textit{No significant overlap in R35 tasking anticipated with this award}\\
%PKH: \$49,455/Total: \$49,455 (1.67 calendar months) \\
%%%%%%%%%%%%%%%%%%%%%%%


%% University of Kentucky asst prof 
1	R56	HL131782	01 (Satin)\dacoi\hfill 09/16-08/17, ($<1$ calendar month)\\
NIH/NHLBI \hfill \$524,989 (incl. indirect)\\
"Monomeric G-protein and cardioprotection from heart failure"\\
The major goal of this project is to model excitation/contraction coupling domain in a transverse tubule dyadic junction. 

%\begin{comment} 
% HIDING BECAUSE INTERNAL 
University of Kentucky, Igniting Research Collaborations Award \dapi \hfill 05/15-08/15 \\
"Simulations of dysregulated intracellular Ca2+-handling in diabetic cardiomyopathy"  \\
PKH: \$25,495 / Total: \$25,495.

University of Kentucky Startup \dapi \hfill 07/01/14-06/30/17 \\
PKH: \$240,000/ Total: \$240,000 (2.0 calendar month)

%\end{comment}

%%%% Postdoc and before 
NIGMS, Competitive Renewal (3 P41GM103426-20)\dasig\hfill 2014 \\
Total: \$1,990,191 

NHLBI, National Research Service Award\dapi  \hfill 2013 \\  % priority score 22
PKH: \$84,000/ Total \$84,000

American Heart Association, Western Affiliates Postdoctoral Fellowship\dapi \hfill 2013 \\
PKH: \$88,000 / Total: \$88,000 

NIGMS,  Supplementary Award  (3 P41 GM103426-19S1)\dasig  \hfill 2012 \\
Total: \$367,613 

% https://www.sbir.gov/sbirsearch/detail/2029
\noindent DoD/Navy, Phase I SBIR \dasig \hfill 2010 \\
"Image Fusion for Submarine Imaging Systems"\\
Total:\$99991 

%Phase I SBIR Automatic Target Recognition (ATR) Algorithm for Submarine Periscope Systems\dacoi (DoD \$100,000) \hfill 2008 \\
% I'm not sure if I helped w writing this Phase I SBIR ... Algorithm for Submarine Periscope Systems\dacoi (DoD \$100,000) \hfill 2008 \\

% http://www.sbir.gov/sbirsearch/detail/2037
DoD, Phase I SBIR \dasig  \hfill 2010 \\
"Investigation of the Debye Effect for Submarine Detection" \\
Total: \$79,995

% http://www.sbir.gov/sbirsearch/detail/96879
%Phase II SBIR Automatic Target Recognition (ATR) Algorithm for Submarine Periscope Systems\dacoi (DoD, \$1,267,015)  \hfill  2009 \\
DoD, Phase II SBIR\dasig \hfill 2009 \\
Algorithm for Submarine Periscope Systems \\
Total: \$1,267,015


%\subsection{Resource proposals}
XSEDE, Supercomputer award, \dapi \hfill 2016\\
Renewal "EF hand calcium binding proteins" \\
PKH: 2M Supercomputer hours

XSEDE, Supercomputer award, \dapi \hfill 2015\\
Renewal "SERCA and its role in cardiac signaling" \\
PKH: 2M Supercomputer hours

XSEDE, Supercomputer award, \dapi \hfill 2014\\
"SERCA and its role in cardiac signaling" \\
PKH: \$148,433 equiv.  


Anton supercomputer award, \dacoi \hfill 2014\\
"Simulations to characterize skeletal muscle Ca2+-binding proteins" 

Anton supercomputer award, \dacoi \hfill 2013 \\
"Microsecond scale simulations to characterize ... full length troponin"

XSEDE, Supercomputer award \dacoi \hfill 2012 

Anton supercomputer award,  \hfill 2011 \\
"Using microsecond scale dynamics to characterize different classes of allosteric interactions" 

Anton supercomputer award,  \dacoi \hfill 2011 \\
"Using microsecond scale dynamics ...  allosteric interactions"

XSEDE,  Startup award for examination of conformational dynamics in SERCA\dapi \hfill 2011 
\newpage


 


%----------------------------------------------------------------------------------------

\vspace{0.2in} % Some whitespace between sections



%----------------------------------------------------------------------------------------

\vspace{0.2in} % Some whitespace between sections

%----------------------------------------------------------------------------------------
%	WORKSHOPS SECTION
%----------------------------------------------------------------------------------------
\newpage
%section{\centerline{TEACHING EXPERIENCE}} 
\section{\centerline{TEACHING EXPERIENCE}} 

\vspace{15pt} % Gap between title and text
\lbi
\item CHE 580: Introduction to computation and modeling of chemical systems, UK, Lexington, KY \hfill 2018-

\item "Introduction to multi-scale modeling", Jilin University, Changchun, China \hfill 2017

\item CHE 446G: Physical Chemistry for Engineers, UK, Lexington, KY \hfill 2016- 

\item "Mathematics of Physical Chemistry Boot Camp", UK, Lexington KY \hfill 2015-

\item CHE 441: Physical Chemistry Lab, UK, Lexington, KY \hfill 2015,17 

\item CHE 105: Gen College Chemistry I, UK, Lexington, KY \hfill 2014-15

\item CHEM 280: Applied Bioinformatics, Guest Lecturer, UCSD, San Diego, CA \hfill 2013 

\item BENG/CHEM 276: Numerical Analysis for Multi-Scale Biology, Guest Lecturer, UCSD, San Diego, CA \hfill 2013

\item Mesoscale Modeling, NBCR Summer Institute, UCSD, San Diego, CA \hfill 2012

\item "Sub-cellular models of calcium diffusion", NBCR Summer Institute, UCSD, San Diego, CA\\ \hfill 2011

\item "Multi-scale Modeling of Cardiac Function", Workshop at International Conference on Biological Physics, San Diego, CA \hfill 2011

\item "Continuum Diffusion in Molecular Systems, NBCR Summer Institute, UCSD, San Diego, CA\\ \hfill 2011 

\item "Special Topics in Signal Processing", Co-lecturer at Arete Associates staff education workshop series, Northridge, CA \hfill 2008 
\lei






%{\sl All courses entailed classroom lecture and lab instruction}

%----------------------------------------------------------------------------------------

\vspace{0.2in} % Some whitespace between sections




%----------------------------------------------------------------------------------------
%	SERVICE SECTION
%----------------------------------------------------------------------------------------

%section{\centerline{SERVICE}} 
\section{\centerline{SERVICE}} 
\vspace{0pt} % Reduce space between section title and contents
{\sl University of Kentucky} \\
%% Committees 
% Univ. and College 
Center of Computational Sciences Faculty Advisory Committee \hfill 2015-present\\
Research/Scholarship Advisory Committee \hfill 2014-present\\
% department
Naff 2016 Symposium Organizer \hfill 2015-2016\\
Graduate Recruiting Committee \hfill 2014-2017\\
Seminar Committee \hfill 2017-\\
Website Committee \hfill 2014-2015 \\
Faculty Advisor to Society of Postdocs \hfill 2014-2016\\ 


%%% Graduate committees
Japheth Gado (Chem E), Thesis Committee \hfill 2018-present \\  
Danielle Schaper (Phys), Thesis Committee \hfill 2017-present \\  
Angela Collier (Phys), Thesis Committee \hfill 2017-present \\ 
Lakshya Malhotra (Phys), Thesis Committee \hfill 2017-present \\
Amira Yu (Chem E, Ph.D.), Thesis Committee \hfill 2017\\
Brandon Franklin (Bio, Ph.D.), Thesis Committee \hfill 2017\\
Wang Hua (Mech E, Ph.D.), Thesis Committee \hfill 2017\\ 
%Understood from grad students advised Bin Sun, Thesis Committee \hfill 2016-present \\     
Joseph Duke (Chem),  Thesis Committee \hfill 2016-present\\
Xiaolu Zhang (Chem), Thesis Committee \hfill 2015\\

%% Outreach
After Hours Residence Life Outreach \hfill 2016\\


{\sl External } \\    
% Outreach 
CREST High School Outreach program \hfill 2016-present\\

% grant and fellowship review review 
National Science Centre - proposal review  \hfill 2018\\
NSF Review Panel  \hfill 2016 (2), 2018 (2) \\
Quarterly XRAC Review Committee \hfill 2015-present\\
Computation Science Graduate Fellowship Screening Committee \hfill 2012-present \\
Petroleum Research Fund proposal review \hfill 2015, 2018 \\

% paper review 
{\sl Manuscripts reviewed } \\    
% 2018
Journal of Computer Aided Molecular Design,
Journal of Physical Chemistry, 
Computers in Biology and Medicine (COMPARE W LAST YEARS CV, SINCE I THINK SOME OF 17s reviewes were done in 18),
Archives of Biochemistry and Biophysics
\hfill 2018
% 2017
PLOS One, 
Scientific Reports,
Journal of Cheminformatics,
Biochemistry (2),
European Biophysics Journal,
eLife,
Mathematical Biosciences,
Biophysical Journal,
Journal of Chemical Physics,
\hfill 2017 \\
% 2016
Biochemistry (2),
Journal of Chemical Physics (3),
PLOS One \hfill 2016 \\
European Biophysics Journal,
Journal of Physical Chemistry B,
Biophysical Journal  \hfill 2015 \\
Journal of Chemical Physics,
Biophysical Journal (2),
FEBS Letters,
Biophysical Journal  \hfill 2014 \\
PNAS \hfill 2013 \\

{\sl Miscellaneous } \\    
Cardiovascular Research Day Poster Judge, 
MACE Symposium Poster Judge \hfill 2018\\
Handling editor for Frontiers Special Topic Issue \hfill 2015 \\
Coordinator of Caltech Alumni Association events in San Diego/Lexington \hfill 2012-present \\
Mini-symposium co-organizer at SIAM Life Sciences meeting, San Diego, CA \hfill 2012 \\
Chaired session at Domain Decomposition Meeting, San Diego, CA \hfill 2011 \\
JAM Steering committee \hfill 2011-2014 \\

\section{\centerline{TRAINING}} 
Center of Research in Obesity and Cardiovascular Disease Monthly Meeting \hfill 2017-present\\
Presentation U! Faculty Fellow, Lexington, KY \hfill 2016\\ 
College of Arts and Sciences Teaching Workshop, Lexington, KY \hfill 2016\\ 
Cottrell Scholars New Faculty Workshop, Washington DC \hfill 2015\\
Center for the Physics of Living Cells Summer School (UIUC) \hfill 2013\\ 
Scientific Ethics (UCSD) \hfill 2013\\ 
College Classroom (Center for Teaching Development, UCSD) \hfill 2013\\
San Diego Lab Management Symposium participant \hfill 2010 \\





\vspace{0.2in} % Some whitespace between sections


%----------------------------------------------------------------------------------------
%	ADVISING    
%----------------------------------------------------------------------------------------

%ection{\centerline{ADVISING}}      
\section{\centerline{ADVISING}}      
\vspace{15pt} % Gap between title and text
{\sl Postdoctoral scholars } \\    
\lbi
\item Caitlin E Scott, Ph.D. \hfill 2014-16 \\
Assistant Professor, Hendrix College\\
\hfill Biophysical Society Travel Award 
\item Selcuk Atalay,  Ph.D.  \hfill 2015-16 
\item Ben Chun, Ph.D. \hfill 2017-present
\lei

{\sl Graduate students } \\    
\lbi
\item Charles Adeniran (CHE) \hfill 2017-2018\\              
\hfill Lyman T Johnson Fellow \hfill 2018 \\
\item Tom Pace (PHY) \hfill 2017-
\item Brad Stewart (CHE) \hfill 2015-2017 \\
\hfill Graduate Teaching award \hfill 2017 \\
\item Bin Sun (CHE) \hfill 2015- \\
\hfill University of Kentucky Graduate Fellowship \hfill 2016\\
\hfill Research Challenge Trust Fund \hfill 2017-2018\\
\hfill Outstanding Performance on the Oral Qualifying Exam \hfill 2017
\lei

{\sl Undergraduate students } \\    
\lbi
\item Amir Kucharski (CHE) \hfill 2014-7 \\
\hfill Gaines Fellowship \\
Admission to WUSTL MD/Ph.D. program
\item Ryan Blood (CME) \hfill 2016-\\
Admission to Notre Dame graduate school \hfill 2018\\
Notebaert Fellow \hfill 2018
\item Andrew Mondragon (CME) \hfill 2017 
\item Dylan Colli (CME) \hfill 2016-\\
\hfill Second place in Graduate Poster Competition AiCHE \hfill 2017\\
\hfill American Heart Association USTiCR fellow \hfill 2018
\item Angela Hinchie (CHE) \hfill 2016\\ 
Admission to University of Pittsburgh graduate school
\item Darin Vaughan (MA,CHE) \hfill 2017- 
\item Rachel Boone (CME) \hfill 2017-\\
\lei

{\sl High school}\\
\lbi
\item Shashank Bhatta (Dunbar High School) \hfill 2017-
\lei



%----------------------------------------------------------------------------------------
%	SOFTWARE SECTION
%----------------------------------------------------------------------------------------

%section{\centerline{PRODUCTS}}
\section{\centerline{PRODUCTS}}
\vspace{0pt} 
\noindent 
\textsc{smolfin} Diffusion-limited association reactions

\textsc{EnzymeKineticsACS} Spatially-decoupled biochemical reactions

\textsc{Smolhomog} Homogenized Smoluchowski solver

\textsc{homogenization} Multi-scale estimates of diffusion tensors

\textsc{sarcomere} Metabolism in half-sarcomere 

\textit{Additional software is available at \href{http://bitbucket.org/huskeypm}{bitbucket.org/huskeypm} and 
\href{https://bitbucket.org/pkhlab/pkh-lab-analyses/}
{bitbucket.org/pkhlab/pkh-lab-analyses/}
}

\vspace{0.2in} % Some whitespace between sections


%----------------------------------------------------------------------------------------

\vspace{0.2in} % Some whitespace between sections


%----------------------------------------------------------------------------------------
%	MEMBERSHIPS SECTION
%----------------------------------------------------------------------------------------

\section{\centerline{MEMBERSHIPS}} 

\vspace{-5pt} % Reduce space between section title and contents

%begin{center}
American Chemical Society\\
Biophysical Society \\
American Heart Association \\
%end{center}

%----------------------------------------------------------------------------------------

\vspace{0.2in} % Some whitespace between sections


%----------------------------------------------------------------------------------------

\vspace{0.2in} % Some whitespace between sections

%----------------------------------------------------------------------------------------
%	PRESENTATIONS       
%----------------------------------------------------------------------------------------
\newpage
%\section{\centerline{INVITED TALKS}} 
\section{\centerline{INVITED TALKS}} 
\vspace{15pt} % Gap between title and text
% perl -ane 'chomp$_; $_=~s/\((\d+)\),\s*//; push(@a,$_." \\hfill $1 \\\\\n") if $_=~/\w+/; print reverse @a if $_=~/Simula Summer/' xxx
\hfill 2018\\
Myofilament Meeting, Madison, WI, 
University of Kentucky (Department of Biomedical Engineering), Lexington KY,
University of Kentucky (Department of Physiology), Lexington KY
Commonwealth Computational Summit, Lexington, KY
Carnegie Mellon/University of Pittsburgh (Dec 2018), Pittsburgh, PA
University of West Virginia, Morgantown, WV (Dec 2018)
\hfill 2017\\
Earlham College, Richmond IN, 
Berea College, Berea, KY, 
Vanderbilt University, Nashville, TN 

\hfill 2016\\
Illinois Institute of Technology, Chicago, IL, 
Rush University, Chicago, IL, 
University of Kentucky (Departments of Math, Physics), Lexington, KY,
University of Missouri, Columbia, MO,
Truman State University, Kirksville, MO,
Tennessee Technical University, Cookesville, TN
%"Thin Filament Regulation: Lessons from Troponin C and other small calcium binding proteins", 
Myofilament Meeting 2016, Madison, WI, 
California Institute of Technology, Pasadena, CA, University of California San Diego, San Diego, CA

\hfill 2015\\
Indiana State University, Terre Haute, IN,
%"Molecular simulation of an integral Ca2+ handling protein: SERCA", 
Simula Summer School, Norway, Oslo,
%"Quantifying the influence of the crowded cytoplasm on small molecule diffusion via homogenization theory", 
Bluegrass Molecular Biophysics Symposium, Lexington, KY,
%"Modeling of calcium signaling in cardiomyocytes", SIAM, 
Salt Lake City, UT

\hfill 2014 \\
University of Kentucky Dept. of Chemical Engineering, Lexington, KY, Furman University, Greenville, SC, 
Oak Ridge National Labs, Oak Ridge, TN, 
Invited Poster at SciMix SERMACs meeting, Nashville, TN,
%"Electrostatic channeling of dihydrofolate in the dihydrofolate reductase/thymidylate synthase complex", 
American Chemical Society National Meeting, Dallas, CA, University of Arizona, Tucson, AZ, 
Loyola University Health Sciences Campus, Chicago, IL, 

\hfill 2013 \\
Northeastern University, Boston, MA, 
%"Open problems in integrative studies of movement in living systems: Molecular to cellular scale modeling of myofibril function", 
University of Washington, Seattle, WA, 
University of North Carolina, Asheville, NC,
%"Homogenization of the Smoluchowski electro-diffusion equation for estimating hindered, anisotropic calcium diffusion in charged environments", 
Fall National ACS meeting, Indianapolis, IN, 
Simula Research Laboratory, Norway, Oslo, 
CVRTI, University of Utah, Salt Lake City, UT, Department of Chemistry, University of Utah, Salt Lake City, UT

\hfill 2011 \\
%"Modeling of Ca2+ signaling in Cardiac Function", 
Gordon Research Seminar on Calcium Signaling, Waterville, ME, 
Mathematics and Biochemistry-Biophysics Seminar at UCSD, San Diego, CA 

%----------------------------------------------------------------------------------------
%	MISC TALKS
%----------------------------------------------------------------------------------------

%section{\centerline{PRESENTATIONS}} 
\newpage
\section{\centerline{PRESENTATIONS}} 
\vspace{15pt} % Gap between title and text
Muscle Forum, University of Kentucky   \hfill 2015 \\

Society of Post-docs, University of Kentucky \hfill 2015 \\


Biophysical Society Annual Meeeting \hfill 2015\\ 

Heart Working Group, University of Kentucky \hfill 2014 \\

Students of the American Chemical Society, University of Kentucky \hfill 2014 \\

"Multi-scale simulations of diffusion-influenced reactions", Poster at Gordon Research Conference, Mount Snow Resort, NH \hfill 2014 \\

"Multi-scale simulations of diffusion-influenced reactions", Talk at William Goddard, III's Birthday Symposium, Pasadena, CA \hfill 2014 \\

"Multi-scale simulations of diffusion-influenced reactions", Poster at ACS National Meeting, Dallas, TX \hfill 2014 \\

"Multi-scale Continuum Modeling and Simulation of Cardiac Function, Talk at Nifty Fifty, Kearny High School, San Diego, CA \hfill 2014 \\

"A Markov-state model for the regulation of the sarcoplasmic reticulum Ca2+ ATPase by phospholamban",  Poster at Biophysical Society Meeting, San Francisco, CA \hfill 2014 \\

"Continuum diffusion: a language for bridging  molecular and cellular scale signaling", Talk at Georgia State University, Atlanta, GA \hfill 2013 \\

"Building a molecular to cellular-scale understanding of Troponin function through simulation", Talk at Ohio State University, Columbus, OH \hfill 2013 \\

"Continuum diffusion: a language for bridging  molecular and cellular scale signaling", Talk at Carnegie Mellon, Pittsburgh, PA \hfill 2013 \\

"Modeling Calcium Dynamics in Realistic Rabbit Ventricular Myocytes with Several Transverse Tubules", Poster at Alternative Muscle Club Meeting, University of California, San Diego  \hfill 2013 \\

"Multi-scale Continuum Modeling and Simulation of Cardiac Function, Talk at Nifty Fifty, Sweetwater High School, El Cajon, CA \hfill 2013 \\

"Substrate association as a two stage process: the diffusional encounter and post-encounter binding", Talk at Modeling Diffusional Encounter and Subsequent Events Mini-Symposium, San Diego, CA  \hfill 2012 \\

"Multi t-tubule modeling: M-times better than a single t-tubule", Talk at Cardiac Physiome Brainstorming session, San Diego, CA  \hfill 2012 \\

"Molecular and sub-cellular modeling of cardiac Troponin C calcium handling", Talk at SIAM Life Sciences Meeting, San Diego, CA  \hfill 2012 \\

"Molecular electrostatics and Diffusion", Talk at NBCR Summer Institute, San Diego, CA \hfill 2012 \\

"High-level science: a dogma for research and employment?", Talk at CSGF Alumni Meeting, Washington DC  \hfill 2012 \\

"Modeling Calcium Dynamics in Realistic Rabbit Ventricular Myocytes with Several Transverse Tubules", Poster at Gordon Conference on Muscle Excitation Contraction, Les Diableret, Switzerland  \hfill 2012 \\

"Stochastic gating regulates calcium association rates in Troponin C and SERCA", Talk at American Chemical Society Meeting, San Diego, CA \hfill 2012 \\

"Molecular and sub-cellular modeling of Ca2+ signaling in cardiomyocytes", Talk for Nifty Fifty, San Diego High School, San Diego, CA  \hfill 2012 \\

"Modeling Calcium Dynamics in Realistic Rabbit Ventricular Myocytes with Several Transverse Tubules", Poster at Biophysical Society Meeting, San Diego, CA  \hfill 2012 \\

"Contributions of structural t-tubule heterogeneities in local Ca2+ signaling in rabbit ventricular myocytes", Poster at NBCR Summer Institute, UCSD, San Diego, CA (Awarded Best Poster) \hfill 2011 \\

"Contributions of structural t-tubule heterogeneities in local Ca2+ signaling  in rabbit ventricular myocytes", Poster at Cardiac Physiome Workshop, Oxford, England \hfill 2011 \\

"Contributions of structural t-tubule heterogeneities in local Ca2+ signaling in rabbit ventricular myocytes", Poster at Gordon Conference on Calcium Signaling, Waterville, ME \hfill 2011 \\

"Accelerated molecular dynamics of sarcoplasmic reticulum  Ca2+ ATPase (SERCA) structural transitions", Poster at International Conference on Biological Physics, San Diego, CA \hfill 2011 \\

"Sub-cellular Ca2+ signaling in cardiac myocytes", Talk at NBCR RAC meeting, UCSD, San Diego, CA \hfill 2011 \\

"Contributions of structural t-tubule heterogeneities and membrane Ca2+ flux localization to local Ca2+ signaling in rabbit ventricular myocytes", Poster at Biophysical Society Meeting, Baltimore, MD  \hfill 2011 \\

"Multi-scale Continuum Modeling and Simulation of Cardiac Function", Talk at Nifty Fifty High School Outreach, Carlsbad, CA \hfill 2011 \\

"Effects of membrane calcium flux localizations and realistic t-tubule geometry on cardiac excitation contraction coupling", Mini-talk at Biological Diffusion and Brownian Dynamics Brainstorm 2 at UCSD, San Diego, CA \hfill 2010 \\



\end{resume} 

%%----------------------------------------------------------------------------------------
%	YOUR NAME AND ADDRESS(ES) SECTION
%----------------------------------------------------------------------------------------

\name{Peter M. Kekenes-Huskey, Ph.D.\\ \\} % Your name at the top

% If you don't want one of the addresses, simply remove all the text in the first or second \address{} bracket
\newcommand{\mytilde}{\raise.17ex\hbox{$\scriptstyle\mathtt{\sim}$}}
\address{{\bf School Address} \\ Department of Chemistry \\ University of Kentucky\\ 219 Chemistry-Physics Building \\Lexington, KY 40506  } % Your address 1

\address{{\bf Contact} \\ 
pkekeneshuskey@uky.edu \\ http://pkh.as.uky.edu \\  (859) 323-9966} % Your address 2

%----------------------------------------------------------------------------------------
\begin{resume}

%%----------------------------------------------------------------------------------------
%%	OBJECTIVE SECTION
%%----------------------------------------------------------------------------------------
%
%\section{\centerline{OBJECTIVE}}
%
%\vspace{8pt} % Gap between title and text
%
%A position in Personnel Administration utilizing skills in recruiting, training and compensation.\\ 

%----------------------------------------------------------------------------------------
%	EDUCATION SECTION
%----------------------------------------------------------------------------------------

\section{\centerline{EDUCATION}} 

\vspace{8pt} % Gap between title and text

{\sl Doctorate of Philosophy}, 
Chemistry  \\ 
California Institute of Technology, Pasadena, CA \hfill  Spring 2009 \\ 
%(GPA: 4.0 in major, 3.40 overall)
 
{\sl Bachelor of Science}, Chemistry \\ 
University of North Carolina, Asheville, NC \hfill May 2001 \\
\emph{Summa Cum Laude} %(GPA: 4.0)

%----------------------------------------------------------------------------------------
 
\vspace{0.2in} % Some whitespace between sections

%----------------------------------------------------------------------------------------
%	PROFESSIONAL EXPERIENCE SECTION
%----------------------------------------------------------------------------------------

\section{\centerline{PROFESSIONAL EXPERIENCE}} 

\vspace{8pt} % Gap between title and text

{\sl University of Kentucky, Lexington, KY } \hfill 2014 - present\\ 
Assistant Professor of Chemistry
%\lbi
%\item Multiscale modeling of cardiac function and diffusional processes
%\lei


{\sl University of California San Diego,  San
Diego CA [JA
McCammon, AD McCulloch] } \hfill 2010 - 2014\\ 
Postdoctoral fellow 
%\lbi
%\item Multiscale modeling of cardiac function and diffusional processes
%\lei

{\sl Arete Associates, Staff Scientist, Northridge CA} \hfill 	2007 - 2010\\
Staff Scientist
%\lbi
%\item Signal processing, detection theory, algorithm design 
%\lei

{\sl Sandia National Laboratory, Albuquerque, NM.  [PS Crozier]} \hfill 	summer 2005\\
Summer Internship
%\lbi
%\item Free energy calculations for ion diffusion in silica nanopores
%\lei

{\sl California Institute of Technology, Pasadena, CA.  [WA Goddard, III]	 } \hfill 2001 - 2007\\
Graduate Student
%\lbi
%\item Drug design algorithm development and application
%\lei
\begin{comment}
{\sl Freie Universitaet zu Berlin, Berlin, Germany.  [EW Knapp]} \hfill 2001 - 2002\\
Fulbright fellow
%\lbi
%\item Molecular dynamics simulation of estrogen derivative binding to the estrogen receptor
%\lei

{\sl U. North Carolina, Asheville, NC.  [G Heard, BE Holmes]	} \hfill 1999 - 2001\\
Undergraduate researcher
%\lbi
%\item Quantum chemistry calculations of chlorofluorocarbon decomposition kinetics
%\lei

{\sl University of Cincinnati, OH.  [T Beck, W Connick]} \hfill 	summer 2000 \\
Summer researcher 
%\lbi
%\item Quantum chemistry calculations of electron transfer processes
%\lei
\end{comment}

%----------------------------------------------------------------------------------------

\vspace{0.2in} % Some whitespace between sections
%----------------------------------------------------------------------------------------
%	AWARDS
%----------------------------------------------------------------------------------------
\section{\centerline{AWARDS}} 
\vspace{8pt} % Gap between title and text



{\sl Faculty}
\begin{itemize} \itemsep -12pt % Reduce space between items
\item  Nominee for \acrfull{uk} Faculty Mentor of the Year \hfill 2018 \\
\item  UK Office of Undergraduate Research's Faculty Mentor of the Week \hfill 2018 \\
\item Doctoral New Investigator Grant from the American Chemical Society \hfill 2017\\
\item UK Arts\&Sciences Award for Innovative Teaching \hfill 2017\\
\item Recognized as "Teacher who made a difference" (UK) \hfill 2016\\
\item UK Nominee for Blavatnik National Awards for Young Scientists \hfill 2016-17\\
\item UK Nominee for 2016 Simon's Investigator of Math
Modeling of Living Systems award \hfill 2015 \\
\end{itemize}

{\sl Post-graduate}
\begin{itemize} \itemsep -12pt % Reduce space between items
\item National Institutes of Health Ruth Kirschstein Postdoctoral Fellow  \hfill 2013 \\
\item American Heart Association Western States Affiliates Postdoctoral
Fellow \hfill 2013 \\
\item Vice President Discretionary Award (Arete Associates)	\hfill 2010 \\
\end{itemize}

{\sl Graduate }
\begin{itemize} \itemsep -12pt % Reduce space between items
\item DOE Computational Science Graduate Fellow	\hfill 2004-2006  \\
\item National Science Foundation Fellow (declined for CSGF, 2003)	\hfill 2002-2003 \\
\item Department of Defense Fellowship (declined for NSF)  \hfill 2001 \\
\item Fulbright Fellow (Germany)	\hfill 2001 \\
\end{itemize}

{\sl Undergraduate}
\begin{itemize} \itemsep -12pt % Reduce space between items
\item Manly Wright Award, Valedictorian of Graduating Class	\hfill 2001 \\
\item Outstanding Senior of the Western Carolinas American Chemical Society	\hfill 2001 \\
\item USA Today All-Academic 3rd Team	\hfill 2001 \\
\item Barry Goldwater Scholar in Science and Mathematics	\hfill 2000-2001 \\
\item W. Carolina ACS Schweizerhalle Scholarship	\hfill 2000-2001 \\
\item Phi Eta Sigma National Honors Fraternity	\hfill 1998 \\
\item Albina Mills Academic Scholarship	\hfill 1998-2001 \\
\item 4th Place UNCA Olivia-Jones Freshman Creative Writing Contest	\hfill 1999 \\
\end{itemize}

{\sl High School}
\begin{itemize} \itemsep -12pt % Reduce space between items
\item Eagle Scout	\hfill 1998 \\
\item Lion's Eye Bank Scholarship for Post-Secondary Education	\hfill 1998 \\
\item Presidential Scholar	\hfill 1998 \\
\item National Honors Society	\hfill 1997 \\
\end{itemize}


\vspace{0.2in} % Some whitespace between sections

%\newpage
%section{\centerline{PUBLICATIONS}}
\vspace{15pt} % Gap between title and text
\section{\centerline{PUBLICATIONS}}
\newcommand{\pmkh}{{\bf P.M. Kekenes-Huskey}}
(8 publications as faculty member of 32 total)
{\sl * signifies equal contribution}

\begin{etaremune} \itemsep 2pt % Reduce space between items
\item Bin Sun, Ryan Blood, Selcuk Atalay, Dylan Colli, Stephen E. Rankin, Barbara L. Knutson and \pmkh, "Simulation-based characterization of electrolyte and small molecule diffusion in imaged oriented mesoporous silica thin films", 
(chemRxiv: \href{https://doi.org/10.26434/chemrxiv.5533066.v1}{5533066}) (submitted)
\item B.D. Stewart, C.E. Scott, G Yin, F Despa, S Despa, and \pmkh, "Characterization of amylin-induced calcium dysregulation in rat ventricular cardiomyocytes", 2017 \arxiv{1704.03353}(submitted)
\item E. C. Cook, B. Sun, \pmkh\ and T.P. Creamer,"Electrostatic Forces Mediate Fast Association of Calmodulin and the Intrinsically Disordered Regulatory Domain of Calcineurin." 2016. \arxiv{1611.04080}
\item JK Siddiqui, SB Tikunova, SD Walton, M Meyer, PP de Tombe, N Neilson, \pmkh, HE Salhi, PML Janssen, BJ Biesiadecki, JP Davis, "Myofilament Calcium Sensitivity:  Consequences of the Effective Concentration of Troponin I," Frontiers in Physiology, 2016, 7:632. \pmid{28066265}
\item A.N. Kucharski, C.E. Scott, J.P. Davis and \pmkh, "Understanding Ion Binding Affinity and Selectivity in $\beta$ Parvalbumin Using Molecular Dynamics and Mean Sphere Approximation Theory," J Phys Chem B, 2016, 120(33):8617-30 \pmid{28066265}
\item \pmkh, C. E. Scott, and S. Atalay, "Quantifying the influence of the crowded cytoplasm on ionic diffusion," J Phys Chem B 2016, 120(33):8696-706 \pmid{27327486}
\item C. E. Scott and \pmkh, "Molecular basis of calcium-induced structural changes of human S100A1," Biophys J, Mar. 2016, 110(5):1052-1063 \pmid{26958883}
\item \pmkh, C. Eun, and A. McCammon, "Enzyme localization, crowding, and buffers collectively modulate diffusion-influenced signal transduction: Insights from continuum diffusion modeling," Journal of Chemical Physics, 2015, 143(9):1-12. \pmid{26342355}
\vspace{10pt}
\end{etaremune}


%----------------------------------------------------------------------------------------

\vspace{0.2in} % Some whitespace between sections

%----------------------------------------------------------------------------------------
%	FUNDING SECTION
%----------------------------------------------------------------------------------------
%\newpage
%section{\centerline{FUNDING}} 
\section{\centerline{FUNDING}} 
\newcommand{\dapi}{$\dag$}
\newcommand{\dacopi}{$^\circ$}
\newcommand{\dacoi}{$^*$}
\newcommand{\dasig}{$^+$}
{\sl \dapi principal investigator}
{\sl \dacopi co-principal investigator}
{\sl \dacoi co-investigator} 
{\sl \dasig significant contributions} 
\vspace{-10pt} % Reduce space between section title and contents

%\subsection{Pending}




(Kekenes-Huskey)\dapi \hfill 09/01/18-08/31/21 (0.40 calendar month)\\
NSF CTMC \hfill \$580,757\\
"CAREER: Unraveling calcium/calmodulin-dependent calcineurin activation via computation"\\
The major goals of this project is to examine aspects of calcium signaling in nanoscale materials. \\

(Brehm, Kekenes-Huskey)\dacopi \hfill 06/01/19-05/31/20 (0.40 calendar month)\\
NASA EPSCoR \hfill \$40,000\\
"Development of a RANS-Based Wall-Model for CartesianGrid Navier-Stokes Solvers"\\
The major goal of this project is to develop an advanced PDE solver amenable to fluid dynamics simulations  \\

%NSF MCB\dacopi \hfill 09/01/17-08/31/21 \\
%"Role of electrostatics in calcineurin activation"\\
%The major goal of this project is to understand the interplay between electrostatics and conformational dynamics in protein/protein interactions\\
%Total: \$711,975 (2.0 calendar month)
% Copied from overleaf cv; merged in completed
\subsection{Active}
\vspace{-10pt}

%%%%%%%%%%%%%%%%%%%%
1 R35 GM124977 (Kekenes-Huskey)\dapi \hfill 09/01/17-08/31/22 (1.89 calendar month) \\
NIH/NIGMS \hfill \$1,558,386.00 (incl. indirect) \\
"Probing cellular intracellular calcium signaling and sensing through computation"\\
The major goals of this project is to develop multi-scale tools to predict intracellular calcium signaling, from single molecules to the cell. %\\
%PKH: \$1,185,641/Total: \$1,185,641 
%%%%%%%%%%%%%%%%5


%%%%%%%%%%%%%%%%%%%%%%%
Petroleum Research Fund (Kekenes-Huskey)\dapi   \hfill 01/01/18-12/31/19 (TBD calendar month)\\
American Chemical Society \hfill \$110,000 (incl. indirects) \\
"Multi-Scale Modeling of Methane Permeation in Defect-Containing Zeolitic Materials"\\
Major goals include developing multi-physical, multi-scale models of gaseous substrates in highly-structured, zeolitic materials. 


5	U01	HL133359	02 (Campbell)\dasig\hfill 08/03/2018-07/31/22\\
NIH/NIGMS \hfill (\$610,274) \\
`Multiscale modeling of inherited cardiomyopathies and therapeutic interventions'\\
The major goal of this project is to create multi-scale models of cardiac function and myopathies, from the molecular to whole-organ levels.  
PKH provides molecular simulation expertise but does not currently draw funds from this award.

%%%%%%%%%%%%%%%%%%%%%%5





\subsection{Completed} 
\vspace{-10pt}

%%%%%%%%%%%%%%%%%%%%%%%
4 P20 GM103527	09 (Cassis)\dacoi \hfill 09/01/17-08/31/20,  (1.67 calendar months) \\
NIH/NIGMS\hfill \$2,257,498\\
50,000 Pilot Support through "Center of Biomedical Research Excellence (COBRE) on Obesity and Cardiovascular Diseases (COCVD)\\
The major goal of this project is to enhance the competitiveness of junior faculty with research programs. 
PKH lab supported through a 50K pilot award. % \\
%\textit{No significant overlap in R35 tasking anticipated with this award}\\
%PKH: \$49,455/Total: \$49,455 (1.67 calendar months) \\
%%%%%%%%%%%%%%%%%%%%%%%


%% University of Kentucky asst prof 
1	R56	HL131782	01 (Satin)\dacoi\hfill 09/16-08/17, ($<1$ calendar month)\\
NIH/NHLBI \hfill \$524,989 (incl. indirect)\\
"Monomeric G-protein and cardioprotection from heart failure"\\
The major goal of this project is to model excitation/contraction coupling domain in a transverse tubule dyadic junction. 

%\begin{comment} 
% HIDING BECAUSE INTERNAL 
University of Kentucky, Igniting Research Collaborations Award \dapi \hfill 05/15-08/15 \\
"Simulations of dysregulated intracellular Ca2+-handling in diabetic cardiomyopathy"  \\
PKH: \$25,495 / Total: \$25,495.

University of Kentucky Startup \dapi \hfill 07/01/14-06/30/17 \\
PKH: \$240,000/ Total: \$240,000 (2.0 calendar month)

%\end{comment}

%%%% Postdoc and before 
NIGMS, Competitive Renewal (3 P41GM103426-20)\dasig\hfill 2014 \\
Total: \$1,990,191 

NHLBI, National Research Service Award\dapi  \hfill 2013 \\  % priority score 22
PKH: \$84,000/ Total \$84,000

American Heart Association, Western Affiliates Postdoctoral Fellowship\dapi \hfill 2013 \\
PKH: \$88,000 / Total: \$88,000 

NIGMS,  Supplementary Award  (3 P41 GM103426-19S1)\dasig  \hfill 2012 \\
Total: \$367,613 

% https://www.sbir.gov/sbirsearch/detail/2029
\noindent DoD/Navy, Phase I SBIR \dasig \hfill 2010 \\
"Image Fusion for Submarine Imaging Systems"\\
Total:\$99991 

%Phase I SBIR Automatic Target Recognition (ATR) Algorithm for Submarine Periscope Systems\dacoi (DoD \$100,000) \hfill 2008 \\
% I'm not sure if I helped w writing this Phase I SBIR ... Algorithm for Submarine Periscope Systems\dacoi (DoD \$100,000) \hfill 2008 \\

% http://www.sbir.gov/sbirsearch/detail/2037
DoD, Phase I SBIR \dasig  \hfill 2010 \\
"Investigation of the Debye Effect for Submarine Detection" \\
Total: \$79,995

% http://www.sbir.gov/sbirsearch/detail/96879
%Phase II SBIR Automatic Target Recognition (ATR) Algorithm for Submarine Periscope Systems\dacoi (DoD, \$1,267,015)  \hfill  2009 \\
DoD, Phase II SBIR\dasig \hfill 2009 \\
Algorithm for Submarine Periscope Systems \\
Total: \$1,267,015


%\subsection{Resource proposals}
XSEDE, Supercomputer award, \dapi \hfill 2016\\
Renewal "EF hand calcium binding proteins" \\
PKH: 2M Supercomputer hours

XSEDE, Supercomputer award, \dapi \hfill 2015\\
Renewal "SERCA and its role in cardiac signaling" \\
PKH: 2M Supercomputer hours

XSEDE, Supercomputer award, \dapi \hfill 2014\\
"SERCA and its role in cardiac signaling" \\
PKH: \$148,433 equiv.  


Anton supercomputer award, \dacoi \hfill 2014\\
"Simulations to characterize skeletal muscle Ca2+-binding proteins" 

Anton supercomputer award, \dacoi \hfill 2013 \\
"Microsecond scale simulations to characterize ... full length troponin"

XSEDE, Supercomputer award \dacoi \hfill 2012 

Anton supercomputer award,  \hfill 2011 \\
"Using microsecond scale dynamics to characterize different classes of allosteric interactions" 

Anton supercomputer award,  \dacoi \hfill 2011 \\
"Using microsecond scale dynamics ...  allosteric interactions"

XSEDE,  Startup award for examination of conformational dynamics in SERCA\dapi \hfill 2011 
\newpage


 

\end{resume} 

% For biosketch
%%\newcommand\person{Dr. Kekenes-Huskey}
\newcommand\person{My}

\newpage
\renewcommand{\thesection}{\Alph{section}}
\section{Personal Statement}
%My research is dedicated toward understanding molecular mechanisms of biochemical signaling pathways through computational modeling. 
%My lab thus studies the interplay between molecular-scale events, such as protein-ligand (or drug) binding, and cellular-scale signaling pathways arising from interactions between proteins, that shape human health. 
These signaling pathways integrate molecular events with biological function and are often optimized to maximize speed, energy efficiency or robustness, through controlling the location, binding kinetics, and molecular composition of participating enzymes and substrates. 
My lab specifically targets \catwo\ signaling in heart cells, which began with successful collaborative studies and grant funding during my postdoctoral studies at the University of California San Diego (UCSD).
Our most recent preliminary work on calcium-regulation in two mice models, for which sarcolemmal (SL) currents are up-regulated, led us to discover the role of increased sarcoplasmic reticulum (SR) calcium load in amplifying cytosolic calcium transients.
We are currently investigating additional cellular mechanisms, by which the cardiomyocyte compensates for increased SL entry of calcium, as well as how to mitigate calcium dysregulation through targeting specific calcium-handling proteins.  
%Having been co-mentored by Andrew McCulloch (UCSD), I am well-trained in working with experimental physiologists, in addition to my strong computational training with Andy McCammon (UCSD) and Bill Goddard (Caltech). 
These projects are done in collaboration with Drs. Jon Satin, Florin Despa and Sanda Despa at the University of Kentucky, who specialize in animal models of cardiac function.
Our long term goal is to leverage these studies to understand how cardiac  signaling pathways are controlled at the cellular level and perturbed in disease, which will strengthen efforts to pharmaceutically restore normal cardiac output.
Two manuscripts nearing submission evidence major contributions from my trainees toward this goal:
\lbn
\item Scott C, Kekenes-Huskey, P. Understanding the structural changes of the calcium-binding human S100A1 protein with molecular dynamics simulations, 
\item Scott C, Atalay S, Kekenes-Huskey P. Quantifying the influence of the crowded cytoplasm on ion and small biomolecule diffusion via homogenization theory
\len

My research is dedicated toward understanding molecular mechanisms of intracellular signaling pathways through computational modeling. 
These signaling pathways integrate molecular events, such as small molecule binding, to orchestrate myriad biological functions; oftentimes, these concerted processes are optimized to maximize speed, energy efficiency or robustness, through controlling the location, binding kinetics, and molecular composition of participating enzymes and substrates. 
My lab specifically targets calcium (\catwo) signaling in mammalian cells with a particularly emphasis on cardiac tissue, beginning with collaborative studies and grant funding during my postdoctoral studies at the University of California San Diego (UCSD) to address some of the most pressing problems in the understanding of cell biology.
Our most recent preliminary work on \catwo-regulation in two murine models, for which sarcolemmal (SL) currents are up-regulated, led us to discover the role of increased sarcoplasmic reticulum (SR) \catwo\ load in amplifying cytosolic \catwo\ transients in those respective phenotypes. 
We are currently investigating additional cellular mechanisms, by which the cardiomyocyte and its structural remodeling compensates for increased SL entry of \catwo\, as well as how to mitigate \catwo\ dysregulation through targeting specific \catwo\-handling proteins. The studies are complemented with molecular simulations of \catwo\ binding protein function as well as mesoscopic ionic transport, which we are used to understand the subcellular basis of signaling at the cellular scale. 
A unique angle to our subcellular approach is our usage of advanced computer vision techniques to process and detect features in bio-imaging data, based on my experience at the defense subcontractor Arete Associates following graduate school. 
Our long-term goal is to leverage these studies to understand how cardiac signaling pathways are controlled at the cellular level and perturbed in disease, which will prioritize molecular strategies to pharmaceutically restore normal cardiac output. 
Our manuscripts 
% 
(\href{https://www.ncbi.nlm.nih.gov/sites/myncbi/1TY9bcXrU0YAs/bibliography/43451609/public/?sort=date&direction=ascending}{MYNBCI}) 
evidence major contributions from my trainees toward this goal:
\lbn
\item Scott C and Kekenes-Huskey P, “Molecular Basis of S100A1 Activation at Saturating and Subsaturating Calcium Concentrations.,” Biophys J, vol. 110, no. 5, pp. 1052–1063, Mar. 2016.
\item Kucharski, N, Scott C, and Kekenes-Huskey P, “Understanding Ion Binding Affinity and Selectivity in β Parvalbumin Using Molecular Dynamics and Mean Sphere Approximation Theory,” J Phys Chem B, Jun. 2016.
\item Atalay S, Scott C, Kekenes-Huskey P. Quantifying the influence of the crowded cytoplasm on ion and small biomolecule diffusion via homogenization theory (in print)
\item Siddiqui, Jm Svetlana B Tikunova, Shane D Walton, Meredith Meyer, Peter P de Tombe, Nathan Neilson, Peter M Kekenes-Huskey, Hussam E Salhi, Paul M L Janssen, Brandon J Biesiadecki, Jonathan P Davis, “Myofilament Calcium Sensitivity:  Consequences of the Effective Concentration of Troponin I" (in revision)
\item Atalay, S, Scott, CE, Satin, J, Kekenes-Huskey, P ,“Microstructure within transverse tubules help control \catwo\-induced \catwo\ release in healthy and pathologically-remodeled cardiomyocytes” (in preparation)
\item Stewart, B, Scott CE, Despa, S, Despa, F Kekenes-Huskey P “Effects of amylin on cardiac calcium homeostatis” (in preparation). 
\item Cook Erik, Sun B, Kekenes-Huskey P, Creamer T “Diffusion limited association of calcineurin and calmodulin” (in preparation). 
\len

It should also be emphasized that I am committed to mentoring undergraduate students through postdoctoral scholars. 
The papers published or in preparation from my lab all have significant contributions from junior scientists at the undergraduate level and up.  
Aside from mentoring projects, we hold weekly group meetings to practice presentation skills and discuss literature, monthly joint group meetings with Dr. Christy Payne, chemical engineering, as well as frequent attendance of seminars on campus. 
I also provide travel support for lab members on at least a yearly basis to gain exposure to other labs’ research approaches and improve networking opportunities. 



\textcolor{red}{WARNING: most recent copy is in faculty docs}
\setcounter{section}{2} 
\section{Contribution to Science} 
\subsection*{Calcium handling proteins}
% Key results, 
% impact on my field of study, 
% my pecific role in the described work
\begin{refsection}
Heart failure is a serious health risk for millions of Americans and is commonly associated with substantial dysregulation of vital \catwo\ signaling pathways.  
Detailed insight into how cardiac proteins regulate \catwo\ signaling therefore offers exciting potential to combat heart disease through protein engineering and drug design.
\person\ research as a postdoc and assistant professor has yielded unprecedented molecular detail into \catwo-binding and sensing mechanisms of several prominent muscle proteins, including troponin, SERCA, parvalbumin and S100A1 \autocite{Lindert2015,Cheng2014,Kekenes-Huskey2012c,Kekenes-Huskey2012a,Lindert2012a,Scott2016,Kucharski2016}.  
These mechanisms provide exquisite control of \catwo\ handling and were unveiled through novel hybrid computational approaches I developed to couple all-atom molecular dynamics simulations with meso-scale Brownian dynamics and statistical mechanics models.
Recently, we have had an exciting development that casts doubt on a popular dogma that \catwo\ binding selectivity is confined to the binding site structure.
Namely, we found that protein internal strain and hydrophobic packing powerfully tune cation binding thermodynamics; this advancement opens the door for new strategies to redesign or potentially (ant)agonize \catwo\ binding proteins to treat diseases accompanied by substantial \catwo\ dysregulation. 

\printbibliography[heading=none] % print section bibliography
\end{refsection}

\subsection*{Signaling in cardiac cells}
% Key results, 
% impact on my field of study, 
% my pecific role in the described work
\begin{refsection}
Of equal importance to the integrity of calcium signaling at the single protein level is the close spatial and temporal coupling between integral calcium-handling proteins. 
Central to maintaining normal \catwo\ homeostasis, for instance, is the close apposition of key ion channels anchored in the cell membrane and sarcoplasmic reticulum; conversely, this coupling is frequently destroyed in later stages of heart disease.
To improve our understanding of the interrelationships between heart cell morphology, intracellular organization and cardiac signaling, I have developed three-dimensional finite element models of \catwo\ \autocite{Hake2012,Kekenes-Huskey2012} and nucleotide signaling \autocite{Kekenes-Huskey2013} that leverage realistic cellular anatomic models from confocal fluorescence and cryoelectron microscopy.
Among the most important discoveries are quantitative estimates of the contributions of protein distribution, 'co-localization' and 'buffering' to maintaining, or desynchronizing, cardiac \catwo\ signaling. 
Presently, we are extending these innovations by integrating advanced computer vision detection algorithms, which could ultimately have translational outcomes in the automated analysis of biopsied tissue. 

%\printbibliography[heading=subbibliography] % print section bibliography
\printbibliography[heading=none] % print section bibliography
\end{refsection}

\subsection*{Multi-scale modeling of coupled biochemical networks}
% Key results, 
% impact on my field of study, 
% my pecific role in the described work
\begin{refsection}
It is well-established for coupled biochemical networks that proteins' spatial organization and effective diffusion rates of their substrates are as important as substrate binding kinetics in controlling intracellular signaling.
Despite the importance of spatial coupling and diffusion, precise characterization of these factors remains a daunting task in real biological systems, given the broad ranges of spatial and temporal scales involved in signaling.
I have advanced the computational biophysics field substantially by drawing on algorithmic approaches, including homogenization theory, to link fine-grained molecular information with micron-scale models of intracellular signaling and metabolic transport.
\autocite{Metzger2014,Eun2014,Eun2013,Kekenes-Huskey2012b,Kekenes-Huskey2014a}.
Particularly exciting are our recent findings \autocite{Kekenes-Huskey2015} that for  signaling processes that exhibit oscillations such as feedback inhibition,  signal frequency and amplitude can be controlled purely by intracellular organization and diffusion rates, without chemical modification of the proteins themselves. 
This raises an intriguing possibility that cells behave like miniature signal processing units that can filter, amplify or even frequency-shift molecular signals. 
Currently we are investigating how cells can adapt signal processing characteristics through morphological changes that reposition key proteins. 


\printbibliography[heading=none] % print section bibliography
\end{refsection}


\end{document}
 % NEED TO COMPILE AS ARTICLE

% NSF 
%

\name{Peter M. Kekenes-Huskey, Ph.D.\\ \\} % Your name at the top

% If you don't want one of the addresses, simply remove all the text in the first or second \address{} bracket
\newcommand{\mytilde}{\raise.17ex\hbox{$\scriptstyle\mathtt{\sim}$}}
\address{{\bf School Address} \\ Department of Chemistry \\ University of Kentucky\\ 219 Chemistry-Physics Building \\Lexington, KY 40506 \\ 
http://pkh.as.uky.edu} % Your address 1

\address{{\bf Permanent Address} \\ 307 Curtin Drive \\ Lexington, KY 40503 \\ pkekeneshuskey@uky.edu \\ (626) 644-9966} % Your address 2

%----------------------------------------------------------------------------------------
\begin{resume}

%%----------------------------------------------------------------------------------------
%%	OBJECTIVE SECTION
%%----------------------------------------------------------------------------------------
%
%\section{\centerline{OBJECTIVE}}
%
%\vspace{8pt} % Gap between title and text
%
%A position in Personnel Administration utilizing skills in recruiting, training and compensation.\\ 

%----------------------------------------------------------------------------------------
%	EDUCATION SECTION
%----------------------------------------------------------------------------------------

\section{\centerline{EDUCATION}} 
\vspace{8pt} % Gap between title and text
{\sl California Institute of Technology, Chemistry } \hfill  Ph.D., 2009 \\ 
{\sl University of North Carolina Asheville, Chemistry} \hfill B.S., 2001 \\

%----------------------------------------------------------------------------------------
 
\vspace{0.2in} % Some whitespace between sections

%----------------------------------------------------------------------------------------
%	PROFESSIONAL EXPERIENCE SECTION
%----------------------------------------------------------------------------------------

\section{\centerline{PROFESSIONAL EXPERIENCE}} 

\vspace{8pt} % Gap between title and text

{\sl Assistant Professor, Chemistry, University of Kentucky} \hfill 2014 - present\\ 
{\sl Postdoctoral fellow, University of California San Diego } \hfill 2010 - 2014\\ 
{\sl Staff Scientist, Arete Associates} \hfill 	2007 - 2010\\
{\sl Intern, Sandia National Laboratory} \hfill 	summer 2005\\
{\sl Graduate Fellow, California Institute of Technology} \hfill 2001 - 2007\\
{\sl Fulbright Fellow, Freie Universitaet zu Berlin} \hfill 2001 - 2002\\
{\sl Undergraduate Researcher, U. North Carolina, Asheville} \hfill 1999 - 2001\\
{\sl Summer researcher, University of Cincinnati} \hfill 	summer 2000 \\

%----------------------------------------------------------------------------------------

\vspace{0.2in} % Some whitespace between sections

%----------------------------------------------------------------------------------------
%	PUBLICATIONS SECTION
%----------------------------------------------------------------------------------------

\section{\centerline{Products}}
Five most relevant publications to this project (Total of 31)
\vspace{15pt} % Gap between title and text
\newcommand{\pmkh}{{\bf P.M. Kekenes-Huskey}}
{\sl * signifies equal contribution}

\begin{etaremune} \itemsep 2pt % Reduce space between items
%\item {\bf P.M.Kekenes-Huskey}, A Edwards, J Hake, A Michailova, AD McCulloch, and JA McCammon, "Effects of t-system microanatomy and membrane transporter arrangement on Ca2+ trigger flux in normal and remodeled rabbit myocytes", (in prep)

\item E. C. Cook, B. Sun, \pmkh\ and T.P. Creamer,"Electrostatic Forces Mediate Fast Association of Calmodulin and the Intrinsically Disordered Regulatory Domain of Calcineurin." 2016. \arxiv{1611.04080}
\item JK Siddiqui, SB Tikunova, SD Walton, M Meyer, PP de Tombe, N Neilson, \pmkh, HE Salhi, PML Janssen, BJ Biesiadecki, JP Davis, "Myofilament Calcium Sensitivity:  Consequences of the Effective Concentration of Troponin I," Frontiers in Physiology, 2016, 7:632. \pmid{28066265}
\item A.N. Kucharski, C.E. Scott, J.P. Davis and \pmkh, "Understanding Ion Binding Affinity and Selectivity in $\beta$ Parvalbumin Using Molecular Dynamics and Mean Sphere Approximation Theory," J Phys Chem B, 2016, 120(33):8617-30 \pmid{28066265}
\item \pmkh, C. E. Scott, and S. Atalay, "Quantifying the influence of the crowded cytoplasm on ionic diffusion," J Phys Chem B 2016, 120(33):8696-706 \pmid{27327486}
\item C. E. Scott and \pmkh, "Molecular basis of calcium-induced structural changes of human S100A1," Biophys J, Mar. 2016, 110(5):1052-1063 \pmid{26958883}

\end{etaremune}

\begin{comment}
\section{\centerline{Other Significant Products}}
Five other significant publications (Total of 30)

\begin{etaremune} \itemsep 2pt % Reduce space between items
\item \pmkh, A. K. Gillette, and J. A. McCammon, "Predicting the influence of long-range molecular interactions on macroscopic-scale diffusion by homogenization of the Smoluchowski equation," The Journal of chemical physics, vol. 140, no. 17, p. 174106, May 2014.
\item C. Eun, \pmkh*, V. T. Metzger, and J. A. McCammon, "A model study of sequential enzyme reactions and electrostatic channeling.," Journal of Chemical Physics, vol. 140, no. 10, pp. 105101-105101, Mar. 2014.
\item \pmkh, T. Liao, A. K. Gillette, J. E. Hake, Y. Zhang, A. P. Michailova, A. D. Mcculloch, and J. A. McCammon, "Molecular and subcellular-scale modeling of nucleotide diffusion in the cardiac myofilament lattice.," Biophys J, vol. 105, no. 9, pp. 2130-2140, Nov. 2013.
\item \pmkh, S. Lindert, and J. McCammon, "Molecular basis of calcium-sensitizing and desensitizing mutations of the human cardiac troponin C regulatory domain: a multi-scale simulation study.," PLOS Computational Biology, vol. 8, no. 11, pp. e1002777-e1002777, Nov. 2012.
\item \pmkh, A. Gillette, J. Hake, and J. A. McCammon, "Finite-element estimation of protein-ligand association rates with post-encounter effects: applications to calcium binding in troponin C and SERCA," Comput Sci Discov, vol. 5, no. 1, p. 014015, 2012.
\end{etaremune}
\end{comment}

%%% FUnding
\newpage
\section{\centerline{FUNDING}} 
\newcommand{\dapi}{$\dag$}
\newcommand{\dacopi}{$^\circ$}
\newcommand{\dacoi}{$^*$}
\newcommand{\dasig}{$^+$}
{\sl \dapi principal investigator}
{\sl \dacopi co-principal investigator}
{\sl \dacoi co-investigator} 
{\sl \dasig significant contributions} 
\vspace{-10pt} % Reduce space between section title and contents

%\subsection{Pending}




(Kekenes-Huskey)\dapi \hfill 09/01/18-08/31/21 (0.40 calendar month)\\
NSF CTMC \hfill \$580,757\\
"CAREER: Unraveling calcium/calmodulin-dependent calcineurin activation via computation"\\
The major goals of this project is to examine aspects of calcium signaling in nanoscale materials. \\

(Brehm, Kekenes-Huskey)\dacopi \hfill 06/01/19-05/31/20 (0.40 calendar month)\\
NASA EPSCoR \hfill \$40,000\\
"Development of a RANS-Based Wall-Model for CartesianGrid Navier-Stokes Solvers"\\
The major goal of this project is to develop an advanced PDE solver amenable to fluid dynamics simulations  \\

%NSF MCB\dacopi \hfill 09/01/17-08/31/21 \\
%"Role of electrostatics in calcineurin activation"\\
%The major goal of this project is to understand the interplay between electrostatics and conformational dynamics in protein/protein interactions\\
%Total: \$711,975 (2.0 calendar month)
% Copied from overleaf cv; merged in completed
\subsection{Active}
\vspace{-10pt}

%%%%%%%%%%%%%%%%%%%%
1 R35 GM124977 (Kekenes-Huskey)\dapi \hfill 09/01/17-08/31/22 (1.89 calendar month) \\
NIH/NIGMS \hfill \$1,558,386.00 (incl. indirect) \\
"Probing cellular intracellular calcium signaling and sensing through computation"\\
The major goals of this project is to develop multi-scale tools to predict intracellular calcium signaling, from single molecules to the cell. %\\
%PKH: \$1,185,641/Total: \$1,185,641 
%%%%%%%%%%%%%%%%5


%%%%%%%%%%%%%%%%%%%%%%%
Petroleum Research Fund (Kekenes-Huskey)\dapi   \hfill 01/01/18-12/31/19 (TBD calendar month)\\
American Chemical Society \hfill \$110,000 (incl. indirects) \\
"Multi-Scale Modeling of Methane Permeation in Defect-Containing Zeolitic Materials"\\
Major goals include developing multi-physical, multi-scale models of gaseous substrates in highly-structured, zeolitic materials. 


5	U01	HL133359	02 (Campbell)\dasig\hfill 08/03/2018-07/31/22\\
NIH/NIGMS \hfill (\$610,274) \\
`Multiscale modeling of inherited cardiomyopathies and therapeutic interventions'\\
The major goal of this project is to create multi-scale models of cardiac function and myopathies, from the molecular to whole-organ levels.  
PKH provides molecular simulation expertise but does not currently draw funds from this award.

%%%%%%%%%%%%%%%%%%%%%%5





\subsection{Completed} 
\vspace{-10pt}

%%%%%%%%%%%%%%%%%%%%%%%
4 P20 GM103527	09 (Cassis)\dacoi \hfill 09/01/17-08/31/20,  (1.67 calendar months) \\
NIH/NIGMS\hfill \$2,257,498\\
50,000 Pilot Support through "Center of Biomedical Research Excellence (COBRE) on Obesity and Cardiovascular Diseases (COCVD)\\
The major goal of this project is to enhance the competitiveness of junior faculty with research programs. 
PKH lab supported through a 50K pilot award. % \\
%\textit{No significant overlap in R35 tasking anticipated with this award}\\
%PKH: \$49,455/Total: \$49,455 (1.67 calendar months) \\
%%%%%%%%%%%%%%%%%%%%%%%


%% University of Kentucky asst prof 
1	R56	HL131782	01 (Satin)\dacoi\hfill 09/16-08/17, ($<1$ calendar month)\\
NIH/NHLBI \hfill \$524,989 (incl. indirect)\\
"Monomeric G-protein and cardioprotection from heart failure"\\
The major goal of this project is to model excitation/contraction coupling domain in a transverse tubule dyadic junction. 

%\begin{comment} 
% HIDING BECAUSE INTERNAL 
University of Kentucky, Igniting Research Collaborations Award \dapi \hfill 05/15-08/15 \\
"Simulations of dysregulated intracellular Ca2+-handling in diabetic cardiomyopathy"  \\
PKH: \$25,495 / Total: \$25,495.

University of Kentucky Startup \dapi \hfill 07/01/14-06/30/17 \\
PKH: \$240,000/ Total: \$240,000 (2.0 calendar month)

%\end{comment}

%%%% Postdoc and before 
NIGMS, Competitive Renewal (3 P41GM103426-20)\dasig\hfill 2014 \\
Total: \$1,990,191 

NHLBI, National Research Service Award\dapi  \hfill 2013 \\  % priority score 22
PKH: \$84,000/ Total \$84,000

American Heart Association, Western Affiliates Postdoctoral Fellowship\dapi \hfill 2013 \\
PKH: \$88,000 / Total: \$88,000 

NIGMS,  Supplementary Award  (3 P41 GM103426-19S1)\dasig  \hfill 2012 \\
Total: \$367,613 

% https://www.sbir.gov/sbirsearch/detail/2029
\noindent DoD/Navy, Phase I SBIR \dasig \hfill 2010 \\
"Image Fusion for Submarine Imaging Systems"\\
Total:\$99991 

%Phase I SBIR Automatic Target Recognition (ATR) Algorithm for Submarine Periscope Systems\dacoi (DoD \$100,000) \hfill 2008 \\
% I'm not sure if I helped w writing this Phase I SBIR ... Algorithm for Submarine Periscope Systems\dacoi (DoD \$100,000) \hfill 2008 \\

% http://www.sbir.gov/sbirsearch/detail/2037
DoD, Phase I SBIR \dasig  \hfill 2010 \\
"Investigation of the Debye Effect for Submarine Detection" \\
Total: \$79,995

% http://www.sbir.gov/sbirsearch/detail/96879
%Phase II SBIR Automatic Target Recognition (ATR) Algorithm for Submarine Periscope Systems\dacoi (DoD, \$1,267,015)  \hfill  2009 \\
DoD, Phase II SBIR\dasig \hfill 2009 \\
Algorithm for Submarine Periscope Systems \\
Total: \$1,267,015


%\subsection{Resource proposals}
XSEDE, Supercomputer award, \dapi \hfill 2016\\
Renewal "EF hand calcium binding proteins" \\
PKH: 2M Supercomputer hours

XSEDE, Supercomputer award, \dapi \hfill 2015\\
Renewal "SERCA and its role in cardiac signaling" \\
PKH: 2M Supercomputer hours

XSEDE, Supercomputer award, \dapi \hfill 2014\\
"SERCA and its role in cardiac signaling" \\
PKH: \$148,433 equiv.  


Anton supercomputer award, \dacoi \hfill 2014\\
"Simulations to characterize skeletal muscle Ca2+-binding proteins" 

Anton supercomputer award, \dacoi \hfill 2013 \\
"Microsecond scale simulations to characterize ... full length troponin"

XSEDE, Supercomputer award \dacoi \hfill 2012 

Anton supercomputer award,  \hfill 2011 \\
"Using microsecond scale dynamics to characterize different classes of allosteric interactions" 

Anton supercomputer award,  \dacoi \hfill 2011 \\
"Using microsecond scale dynamics ...  allosteric interactions"

XSEDE,  Startup award for examination of conformational dynamics in SERCA\dapi \hfill 2011 
\newpage




\section{\centerline{SYNERGISTIC ACTIVITIES}} 
\vspace{8pt} % Gap between title and text

{\sl Courses Taught: General Chemistry, Physical Chemistry for Engineers, Physical Chemistry Lab} \hfill 2014 - present\\ 
{\sl Manuscripts reviewed (10+) for several journals} \hfill 2014 - present\\ 
{\sl Proposals reviewed for NSF, ACS} \hfill 2015 - present\\ 
{\sl Standing member of NSF XSEDE Review Panel} \hfill 2015 - present\\ 
{\sl Standing member of DOE CS Graduate Fellowship Review Committee} \hfill 2012 - present\\ 




\vspace{0.2in} % Some whitespace between sections
\end{resume} 


\end{document}
