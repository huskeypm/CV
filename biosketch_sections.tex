%\newcommand\person{Dr. Kekenes-Huskey}
\newcommand\person{My}

\newpage
\renewcommand{\thesection}{\Alph{section}}
\section{Personal Statement}
%My research is dedicated toward understanding molecular mechanisms of biochemical signaling pathways through computational modeling. 
%My lab thus studies the interplay between molecular-scale events, such as protein-ligand (or drug) binding, and cellular-scale signaling pathways arising from interactions between proteins, that shape human health. 
These signaling pathways integrate molecular events with biological function and are often optimized to maximize speed, energy efficiency or robustness, through controlling the location, binding kinetics, and molecular composition of participating enzymes and substrates. 
My lab specifically targets \catwo\ signaling in heart cells, which began with successful collaborative studies and grant funding during my postdoctoral studies at the University of California San Diego (UCSD).
Our most recent preliminary work on calcium-regulation in two mice models, for which sarcolemmal (SL) currents are up-regulated, led us to discover the role of increased sarcoplasmic reticulum (SR) calcium load in amplifying cytosolic calcium transients.
We are currently investigating additional cellular mechanisms, by which the cardiomyocyte compensates for increased SL entry of calcium, as well as how to mitigate calcium dysregulation through targeting specific calcium-handling proteins.  
%Having been co-mentored by Andrew McCulloch (UCSD), I am well-trained in working with experimental physiologists, in addition to my strong computational training with Andy McCammon (UCSD) and Bill Goddard (Caltech). 
These projects are done in collaboration with Drs. Jon Satin, Florin Despa and Sanda Despa at the University of Kentucky, who specialize in animal models of cardiac function.
Our long term goal is to leverage these studies to understand how cardiac  signaling pathways are controlled at the cellular level and perturbed in disease, which will strengthen efforts to pharmaceutically restore normal cardiac output.
Two manuscripts nearing submission evidence major contributions from my trainees toward this goal:
\lbn
\item Scott C, Kekenes-Huskey, P. Understanding the structural changes of the calcium-binding human S100A1 protein with molecular dynamics simulations, 
\item Scott C, Atalay S, Kekenes-Huskey P. Quantifying the influence of the crowded cytoplasm on ion and small biomolecule diffusion via homogenization theory
\len

My research is dedicated toward understanding molecular mechanisms of intracellular signaling pathways through computational modeling. 
These signaling pathways integrate molecular events, such as small molecule binding, to orchestrate myriad biological functions; oftentimes, these concerted processes are optimized to maximize speed, energy efficiency or robustness, through controlling the location, binding kinetics, and molecular composition of participating enzymes and substrates. 
My lab specifically targets calcium (\catwo) signaling in mammalian cells with a particularly emphasis on cardiac tissue, beginning with collaborative studies and grant funding during my postdoctoral studies at the University of California San Diego (UCSD) to address some of the most pressing problems in the understanding of cell biology.
Our most recent preliminary work on \catwo-regulation in two murine models, for which sarcolemmal (SL) currents are up-regulated, led us to discover the role of increased sarcoplasmic reticulum (SR) \catwo\ load in amplifying cytosolic \catwo\ transients in those respective phenotypes. 
We are currently investigating additional cellular mechanisms, by which the cardiomyocyte and its structural remodeling compensates for increased SL entry of \catwo\, as well as how to mitigate \catwo\ dysregulation through targeting specific \catwo\-handling proteins. The studies are complemented with molecular simulations of \catwo\ binding protein function as well as mesoscopic ionic transport, which we are used to understand the subcellular basis of signaling at the cellular scale. 
A unique angle to our subcellular approach is our usage of advanced computer vision techniques to process and detect features in bio-imaging data, based on my experience at the defense subcontractor Arete Associates following graduate school. 
Our long-term goal is to leverage these studies to understand how cardiac signaling pathways are controlled at the cellular level and perturbed in disease, which will prioritize molecular strategies to pharmaceutically restore normal cardiac output. 
Our manuscripts 
% 
(\href{https://www.ncbi.nlm.nih.gov/sites/myncbi/1TY9bcXrU0YAs/bibliography/43451609/public/?sort=date&direction=ascending}{MYNBCI}) 
evidence major contributions from my trainees toward this goal:
\lbn
\item Scott C and Kekenes-Huskey P, “Molecular Basis of S100A1 Activation at Saturating and Subsaturating Calcium Concentrations.,” Biophys J, vol. 110, no. 5, pp. 1052–1063, Mar. 2016.
\item Kucharski, N, Scott C, and Kekenes-Huskey P, “Understanding Ion Binding Affinity and Selectivity in β Parvalbumin Using Molecular Dynamics and Mean Sphere Approximation Theory,” J Phys Chem B, Jun. 2016.
\item Atalay S, Scott C, Kekenes-Huskey P. Quantifying the influence of the crowded cytoplasm on ion and small biomolecule diffusion via homogenization theory (in print)
\item Siddiqui, Jm Svetlana B Tikunova, Shane D Walton, Meredith Meyer, Peter P de Tombe, Nathan Neilson, Peter M Kekenes-Huskey, Hussam E Salhi, Paul M L Janssen, Brandon J Biesiadecki, Jonathan P Davis, “Myofilament Calcium Sensitivity:  Consequences of the Effective Concentration of Troponin I" (in revision)
\item Atalay, S, Scott, CE, Satin, J, Kekenes-Huskey, P ,“Microstructure within transverse tubules help control \catwo\-induced \catwo\ release in healthy and pathologically-remodeled cardiomyocytes” (in preparation)
\item Stewart, B, Scott CE, Despa, S, Despa, F Kekenes-Huskey P “Effects of amylin on cardiac calcium homeostatis” (in preparation). 
\item Cook Erik, Sun B, Kekenes-Huskey P, Creamer T “Diffusion limited association of calcineurin and calmodulin” (in preparation). 
\len

It should also be emphasized that I am committed to mentoring undergraduate students through postdoctoral scholars. 
The papers published or in preparation from my lab all have significant contributions from junior scientists at the undergraduate level and up.  
Aside from mentoring projects, we hold weekly group meetings to practice presentation skills and discuss literature, monthly joint group meetings with Dr. Christy Payne, chemical engineering, as well as frequent attendance of seminars on campus. 
I also provide travel support for lab members on at least a yearly basis to gain exposure to other labs’ research approaches and improve networking opportunities. 



\textcolor{red}{WARNING: most recent copy is in faculty docs}
\setcounter{section}{2} 
\section{Contribution to Science} 
\subsection*{Calcium handling proteins}
% Key results, 
% impact on my field of study, 
% my pecific role in the described work
\begin{refsection}
Heart failure is a serious health risk for millions of Americans and is commonly associated with substantial dysregulation of vital \catwo\ signaling pathways.  
Detailed insight into how cardiac proteins regulate \catwo\ signaling therefore offers exciting potential to combat heart disease through protein engineering and drug design.
\person\ research as a postdoc and assistant professor has yielded unprecedented molecular detail into \catwo-binding and sensing mechanisms of several prominent muscle proteins, including troponin, SERCA, parvalbumin and S100A1 \autocite{Lindert2015,Cheng2014,Kekenes-Huskey2012c,Kekenes-Huskey2012a,Lindert2012a,Scott2016,Kucharski2016}.  
These mechanisms provide exquisite control of \catwo\ handling and were unveiled through novel hybrid computational approaches I developed to couple all-atom molecular dynamics simulations with meso-scale Brownian dynamics and statistical mechanics models.
Recently, we have had an exciting development that casts doubt on a popular dogma that \catwo\ binding selectivity is confined to the binding site structure.
Namely, we found that protein internal strain and hydrophobic packing powerfully tune cation binding thermodynamics; this advancement opens the door for new strategies to redesign or potentially (ant)agonize \catwo\ binding proteins to treat diseases accompanied by substantial \catwo\ dysregulation. 

\printbibliography[heading=none] % print section bibliography
\end{refsection}

\subsection*{Signaling in cardiac cells}
% Key results, 
% impact on my field of study, 
% my pecific role in the described work
\begin{refsection}
Of equal importance to the integrity of calcium signaling at the single protein level is the close spatial and temporal coupling between integral calcium-handling proteins. 
Central to maintaining normal \catwo\ homeostasis, for instance, is the close apposition of key ion channels anchored in the cell membrane and sarcoplasmic reticulum; conversely, this coupling is frequently destroyed in later stages of heart disease.
To improve our understanding of the interrelationships between heart cell morphology, intracellular organization and cardiac signaling, I have developed three-dimensional finite element models of \catwo\ \autocite{Hake2012,Kekenes-Huskey2012} and nucleotide signaling \autocite{Kekenes-Huskey2013} that leverage realistic cellular anatomic models from confocal fluorescence and cryoelectron microscopy.
Among the most important discoveries are quantitative estimates of the contributions of protein distribution, 'co-localization' and 'buffering' to maintaining, or desynchronizing, cardiac \catwo\ signaling. 
Presently, we are extending these innovations by integrating advanced computer vision detection algorithms, which could ultimately have translational outcomes in the automated analysis of biopsied tissue. 

%\printbibliography[heading=subbibliography] % print section bibliography
\printbibliography[heading=none] % print section bibliography
\end{refsection}

\subsection*{Multi-scale modeling of coupled biochemical networks}
% Key results, 
% impact on my field of study, 
% my pecific role in the described work
\begin{refsection}
It is well-established for coupled biochemical networks that proteins' spatial organization and effective diffusion rates of their substrates are as important as substrate binding kinetics in controlling intracellular signaling.
Despite the importance of spatial coupling and diffusion, precise characterization of these factors remains a daunting task in real biological systems, given the broad ranges of spatial and temporal scales involved in signaling.
I have advanced the computational biophysics field substantially by drawing on algorithmic approaches, including homogenization theory, to link fine-grained molecular information with micron-scale models of intracellular signaling and metabolic transport.
\autocite{Metzger2014,Eun2014,Eun2013,Kekenes-Huskey2012b,Kekenes-Huskey2014a}.
Particularly exciting are our recent findings \autocite{Kekenes-Huskey2015} that for  signaling processes that exhibit oscillations such as feedback inhibition,  signal frequency and amplitude can be controlled purely by intracellular organization and diffusion rates, without chemical modification of the proteins themselves. 
This raises an intriguing possibility that cells behave like miniature signal processing units that can filter, amplify or even frequency-shift molecular signals. 
Currently we are investigating how cells can adapt signal processing characteristics through morphological changes that reposition key proteins. 


\printbibliography[heading=none] % print section bibliography
\end{refsection}


\end{document}
